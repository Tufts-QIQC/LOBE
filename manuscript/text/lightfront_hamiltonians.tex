\section{Quantum Field Theory Hamiltonians}
\label{subsec:qft-hamiltonians}

The two quantum field theory models studied in section \ref{sec:results} are the $\phi^4$ and Yukawa model.
In performing a Legendre transformation from the Lagrangian to the Hamiltonian, $\mathcal{L} \rightarrow H$, one must choose an explicit coordinate system.
Any field theory calculation done in a Hamiltonian approach benefits from using \textit{front form} (lightfront) coordinates \cite{Dirac1949}.
Lightfront coordinates are defined as $x^\mu = \left(x^+, x^-, x^1, x^2 \right)$, where $x^+ \equiv x^0 + x^3$ and $x^- \equiv x^0 - x^3$.
$x^+$ is the `\textit{time-like}' coordinate, while $x^-$ is the `\textit{space-like}' coordinate.
In each model studied above, the transverse spacial coordinates $x^1, x^2$ are discarded, and the only independent coordinates are $x^+$ and $x^-$.

One main benefit of using this set of coordinates is that the Hamiltonian eigenvalue equation, whose eigenvectors are bound states of the theory, is simpler than in instant (standard) coordinates.
This is because the Hamiltonian operator in instant form coordinates is $H = \sqrt{\vec{P}^2 - m^2}$, where the presence of the square root leads to an ambiguity.
In front form coordinates, the Hamiltonian operator is given via the lightfront `energy' $\hat P^- = \frac{\left(\vec{P}^\perp\right)^2 + m^2}{P^+}$, where the square root is absent.

In one lightfront space and time dimension, the Hamiltonian eigenvalue equation is given by $\hat P^-|\psi\rangle = \frac{m^2}{P^+}|\psi\rangle$.
Any state can be expanded as $$\ket{\psi} = \sum_n \int d[\mu_n] \ket{\mu_n}\braket{\mu_n}{\psi},$$ where $\{\ket{\mu_n}\}$ is a set of Fock states with $n$ particles, i.e. $\ket{f}, \ket{ff}, \ket{b}, \ket{fb}, \dots$.

The lightfront `momentum' $P^+$ is a conserved quantity which every constituent's longitudinal momentum in a given Fock state $\in \{\ket{\mu_n}\}$ must sum to $P^+$.
The `modes' of the ladder operators correspond to discrete longitudinal momentum quantum numbers $k$, where the discretized approximation to a continuous constituent momentum $p^+ = \frac{\pi k}{L}$, where $L$ truncates $x^-$ to a finite region.
Thus, the conservation of $P^+$ is analogous to $\sum k = K$, labeled the \textit{resolution} \cite{}.
Increasing the resolution leads to a larger Hilbert space of states, and thus the number of terms in the Hamiltonian that act non-trivially on this set of states also expands.
This is why increasing $K$ is a metric for `number of terms in the Hamiltonian'.


% Here, the Yukawa and $\phi^4$ Hamiltonians will be written out in lightfront coordinates in one space ($x^-$) and one time ($x^+$) dimension.
% In the continuum, the Hamiltonian, which is derived from the Lagrangian, can be written as integrals of the relevant fields over longitudinal momentum. 
% In the discrete case, the momentum is discretized, and thus integrals become sums.
% Each model will have a different $\mathcal{H}$, depending on the fields involved and the interaction term.

% Each model is composed of fields with discrete momenta $k \in \{1/2, 3/2, \dots\}$ for fermions or $k \in \{1, 2, 3, \dots\}$ for bosons due to their symmetry properties. 
% The discretized fields are given as:

% \begin{equation}
%     \label{eq:psi_discrete}
%     \psi(q) = 2L \frac{\theta(k)u(p_k^+)b_k+ \theta(-k)v(p_{-k}^+)d_{-k}^\dagger}{\sqrt{4\pi \abs{k}}}
% \end{equation}
% \begin{equation}
%     \label{eq:psibar_discrete}
%     \bar \psi(q) = 2L \frac{\theta(k)\bar u(p_k^+)b_k^\dagger + \theta(-k)\bar v(p_{-k}^+)d_{-k}}{\sqrt{4\pi \abs{k}}}
% \end{equation}
% \begin{equation}
%     \label{eq:phi_discrete}
%     \phi(x) = 2L \frac{\theta(k)a_k + \theta(-k)a_{-k}^\dagger}{\sqrt{4\pi \abs{k}}}
% \end{equation}
% The discretized momentum is given by $p_k^+ = \frac{2\pi}{L}k$. 
% In the discretized Hamiltonians below, the momentum sums run to a value $K$, known as the \textit{resolution}.
% This is analogous to the mode cutoff discribed throughout the text. 
% Increasing the resolution $K$ recovers the continuum limit.

% The spinors $u, v$ and gamma matrices $\gamma^+$ have the following definitions in 1+1D lightfront coordinates:

% \begin{align*}
%     u(p_k^+) = \frac{1}{\sqrt{p_k^+}}\left[\begin{matrix} p^+ \\ m \end{matrix}\right]\\
%     v(p_k^+) = \frac{1}{\sqrt{p_k^+}}\left[\begin{matrix} -p^+ \\ m \end{matrix}\right]\\
%     \gamma^+ = \left[\begin{matrix} 0 & 1 \\ 1 & 0 \end{matrix}\right]
% \end{align*}


% \subsection{Yukawa}
% In the Yukawa model, the discrete Hamiltonian is a sum of three terms: $H = H_0 + H_{\text{3-point}} + H_{\text{inst.}}$.
% The free part of the Hamiltonian is 

% \begin{equation}
%     H_0 = \frac{L}{2\pi}\sum_{k = 1/2}^K \frac{m_f^2}{p_k^+}\left(b_k^\dagger b_k + d_k^\dagger d_k \right) + \frac{L}{2\pi}\sum_{k = 1}^K \frac{m_b^2}{p_k^+}a_k^\dagger a_k 
% \end{equation}

% The 3-body interaction piece comes from a sum over the fields $\bar \psi \psi \phi$:

% \begin{equation}
%     H_{\text{3-body}} = \frac{2Lg}{(2L)^3}\sum_{\substack{k_1, k_2 \in \mathbb{Z} + \frac{1}{2} \\ k_3 \in \mathbb{Z} \setminus \{0\}}}
%     \delta_{k1, k2 + k3}:\bar \psi(k_1)\psi(k_2)\phi(k_3):.
% \end{equation}

% Lastly, the instantaneous interaction, arising from constraints on the spinor fields in lightfront coordinates can be written in terms of fields as:
% \begin{equation}
%     H_{\text{inst.}} = \frac{2Lg^2}{(2L)^4}\sum_{\substack{k_1, k_4 \in \mathbb{Z} + \frac{1}{2} \\ k_2, k_3 \in \mathbb{Z} \setminus \{0\}}}
%     \delta_{k1, k2 + k3 + k4}:\bar \psi(k_1)\phi(k_2)\frac{\gamma^+}{k_3 + k_4}\phi(k_3)\psi(k_4):.
% \end{equation}

% The explicit form of these Hamiltonians written in the manner of equation \ref{eq:lco} can be found by expanding the fields in the forms found above.

% \subsection{$\phi^4$}
% The $\phi^4$ Hamiltonian can be written out in lightfront coordinates, similar to Yukawa theory. Using the same field given in equation \ref{eq:phi_discrete}, the discrete Hamiltonian is


% \begin{align}
%     H = 
% \sum_{k_1 = 1}^{K} \frac{m_B^2}{p_{k_1}^+} a_1^\dagger a_1
% &+
% 4 g
% \frac{1}{2L} \sum_{k_1, k_2, k_3, k_4 = 1}^{K}
% \frac{1}{\sqrt{p_{k_1}^+ p_{k_2}^+ p_{k_3}^+ p_{k_4}^+}}
% \delta_{k_1 + k_2 + k_3, k_4}
% \left(
%   a_1^\dagger a_2^\dagger a_3^\dagger a_4
% + a_4^\dagger a_3 a_2 a_1
% \right)
% \\ 
% & +6 g
% \frac{1}{2L} \sum_{k_1, k_2, k_3, k_4 = 1}^{K}
% \frac{1}{\sqrt{p_{k_1}^+ p_{k_2}^+ p_{k_3}^+ p_{k_4}^+}}
% \delta_{k_1 + k_2, k_3 + k_4}
% a_1^\dagger a_2^\dagger a_3 a_4 \\ \nonumber
% \end{align}