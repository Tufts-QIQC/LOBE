\section{Introduction}
\label{sec:intro}

The simulation of many-body quantum systems is a promising potential application for quantum computers \cite{feynman2018simulating}.
In digital quantum simulation, quantum algorithms are composed using a series of unitary operations.
Accessing the information of non-unitary operators - such as the Hamiltonian or other Hermitian observables of a quantum system - within a quantum algorithm is a necessary subroutine for conducting such simulations.
This encoding task has been pursued through various means, resulting in methods such as Trotterization \cite{suzuki1976generalized,hatano2005finding,lie1893theorie,trotter1959product,childs2021theory} and Block-Encoding \cite{lin2022lecture, poulin2018quantum, low2019hamiltonian}.

Block-Encoding describes a general strategy for encoding a non-unitary matrix within a chosen subspace (block) of a larger unitary operator.
Two general frameworks for constructing block-encodings of different operators - sparse block-encodings \cite{berry2009black, childs2009universal, lin2022lecture} and Linear Combinations of Unitaries (LCU) \cite{childs2012hamiltonian} - have allowed for the exploration of explicitly compiled block-encodings of several systems.

Understanding the spacetime quantum resources - the number of qubits (space), the number of operations (time), and the rescaling factor - required for quantum simulation algorithms is important for understanding the feasability of simulating different systems.
These quantum resource estimates are crucial as they allow us to gauge the practical usefulness of quantum computers, particularly those that have experimentally demonstrated quantum error correction \cite{bluvstein2024logical, acharya2024quantum}.

Many previous works have numerically investigated quantum simulation algorithms for purely fermionic systems, with a particular emphasis on the simulation of molecules in quantum chemistry \cite{aspuru2005simulated, peruzzo2014variational, babbush2014adiabatic, o2016scalable, babbush2018encoding, google2020hartree, lee2021even, kivlichan2020improved, campbell2021early}.
However, an alternative set of interesting quantum systems to simulate are those that are derived from quantum field theories \cite{Peskin:1995ev, jordan2012quantum} \ws{@Gus, pls add citation for all models (quartic, static, phi, yukawa)}, which have applications in areas such as high-energy physics \cite{bauer2023quantum}.
These systems often include interactions between fermions, antifermions, and bosons and several works have produced quantum resource estimates for simulating such systems \cite{camps2024explicit, liu2024efficient, rhodes2024exponential}.

In this work, we provide a framework for constructing block-encodings of second-quantized operators, which we refer to as Ladder Operator Block-Encoding (LOBE).
This framework does not require the use of operator transformations that expand fermionic \cite{jordan1928paulische, bravyi2002fermionic, seeley2012bravyi} and bosonic \cite{somma2005quantum,standard-binary} ladder operators in the Pauli operator basis.
Therefore, LOBE avoids the potential overhead caused by these operator transformations by directly block-encoding the operators in their second-quantized forms.

We give numerical quantum resource estimates for implementing block-encodings of several classes of operators and several Hamiltonians that arise in quantum field theories.
This includes purely fermionic systems, purely bosonic systems, and systems which include fermions, antifermions, and bosons. 
We analyze the numerical spacetime quantum resources for LOBE - in comparison to techniques which require mapping ladder operators onto the Pauli basis - and find that LOBE results in constructions with lower numerical quantum resources and favorable scaling with respect to several parameters for many of the systems examined.

This work is organized as follows.
In Section \ref{sec:theory}, we review the action of ladder operators on quantum states in second-quantization.
In Section \ref{sec:block-encoding}, we review block-encodings and discuss frameworks for constructing block-encodings of different operators.
In Section \ref{sec:ladder-op-oracles}, we describe the LOBE framework, show compiled block-encodings for several classes of second-quantized operators, and give analytical spacetime costs of the associated constructions.
In Section \ref{sec:results}, we provide numerical quantum resource estimates for block-encodings of various classes of operators and Hamiltonians.
In Section \ref{sec:conclusions}, we summarize the results presented in this work and discuss future directions.
Additionally, a glossary that defines the terminology and variables used throughout this work is given in Appendix \ref{sec:glossary}.
