\section{Quantum Field Theory Hamiltonians}
\label{subsec:qft-hamiltonians}

The two quantum field theory models studied in section \ref{sec:results} are the $\phi^4$ and Yukawa model.
In performing a Legendre transformation from the Lagrangian to the Hamiltonian, $\mathcal{L} \rightarrow H$, one must choose an explicit coordinate system.
Any field theory calculation done in a Hamiltonian approach benefits from using \textit{front form} (lightfront) coordinates \cite{Dirac1949}.
Lightfront coordinates are defined as $x^\mu = \left(x^+, x^-, x^1, x^2 \right)$, where $x^+ \equiv x^0 + x^3$ acts as the space coordinate and $x^- \equiv x^0 - x^3$ acts as the time coordinate.
In each model studied above, the transverse spacial coordinates $x^1, x^2$ are discarded, and the only independent coordinates are $x^+$ and $x^-$.

One main benefit of using this set of coordinates is that the Hamiltonian eigenvalue equation, whose eigenvectors are bound states of the theory, is simpler than in instant (standard) coordinates.
This is because the Hamiltonian operator in instant form coordinates is $H = \sqrt{\vec{P}^2 - m^2}$, where the presence of the square root leads to an ambiguity.
In front form coordinates, the Hamiltonian operator is given via the lightfront `energy' $\hat P^- = \frac{\left(\vec{P}^\perp\right)^2 + m^2}{P^+}$, where the square root is absent.

In one lightfront space and time dimension, the Hamiltonian eigenvalue equation is given by $\hat P^-|\psi\rangle = \frac{m^2}{P^+}|\psi\rangle$.
Any state can be expanded as $$\ket{\psi} = \sum_n \int d[\mu_n] \ket{\mu_n}\braket{\mu_n}{\psi},$$ where $\{\ket{\mu_n}\}$ is a set of Fock states with $n$ particles, i.e. $\ket{f}, \ket{ff}, \ket{b}, \ket{fb}, \dots$.

The lightfront `momentum' $P^+$ is a conserved quantity which every constituent's longitudinal momentum in a given Fock state $\in \{\ket{\mu_n}\}$ must sum to $P^+$.
The `modes' of the ladder operators correspond to discrete longitudinal momentum quantum numbers $k$, where the discretized approximation to a continuous constituent momentum $p^+ = \frac{\pi k}{L}$, where $L$ truncates $x^-$ to a finite region.
Thus, the conservation of $P^+$ is analogous to $\sum k = K$, labeled the \textit{resolution} \cite{}.
Increasing the resolution leads to a larger Hilbert space of states, and thus the number of terms in the Hamiltonian that act non-trivially on this set of states also expands.
This is why increasing $K$ is a metric for `number of terms in the Hamiltonian'.


