\section{Glossary}
\label{sec:glossary}

In this Section, we explicitly define some of the technical phrases and symbols used throughout this work.

\begin{itemize}
    \item \textit{term} ($T$): An operator defined as a product of ladder operators.
    \item \textit{Active Mode}: A fermionic or bosonic mode upon which a ladder operator is being non-trivially applied. 
    \item \textit{block-encoding ancillae}: A register of qubits that give additional degrees of freedom to produce a block-encoding for the operator in a larger Hilbert space.
    \item \textit{"zeroed-out"}: When the coefficient of a state becomes zero due to the application of an operator.
    \item \textit{encoded subspace}: The chosen subspace of the block-encoding ancillae that denotes the subspace in which the non-unitary operator is encoded. Typically, this is the subspace where all block-encoding ancillae are in the $\ket{0}$ state.
    \item \textit{clean ancillae}: A register of qubits that are promised to begin in the $\ket{0}$ and are returned to the $\ket{0}$ state at the end of a particular operation. 
    \item \textit{all-zero state}: A state of a register where all qubits are in the $\ket{0}$ state.
    \item $L$: The number of terms in the Hamiltonian.
    \item $\alpha_l$: The coefficient of the $l^\text{th}$ term in a linear combination. Assumed to be real and positive unless otherwise stated.
    \item $\Omega$: The occupation cutoff for the bosonic modes. $\Omega$ gives the maximum number of bosons that can be present in a single mode. 
    \item $I$: The number of fermionic or bosonic modes. The subscripts $b$ and $a$ will be used to denote the number of fermionic and bosonic modes respectively.
    \item $b_i$: Fermionic annihilation (creation - $b_i^\dagger$) operator acting on mode $i$.
    \item $d_i$: Antifermionic annihilation (creation - $d_i^\dagger$) operator acting on mode $i$.
    \item $a_i$: Bosonic annihilation (creation - $a_i^\dagger$) operator acting on mode $i$.
    \item $n_{i_b}$: The number of fermions ($b$) occupying the $i^{th}$ mode. $n_{i_d}$ and $n_{i_a}$ give the occupancy for antifermions and bosons respectively.
    \item $\omega_{i}$: The number of bosons occupying the $i^{th}$ bosonic mode.
    \item $N_A$: The dimension of a matrix $A$.
    \item $\beta_\psi$: The amplitude of a state that is outside of the encoded subspace after a block-encoding unitary is applied to $\ket{\psi}$.
    \item $\lambda$: The rescaling factor imposed on the operator for a given block-encoding.
    \item $w$: The index of the $w^\text{th}$ least-significant bit in a binary encoding.
    \item $B$: The number of active modes within a term. The  with ladder operators acting on them within the term $T_l$.
    \item $C$: The number of active modes within a term \textit{excluding} modes upon which a number operator is being applied.
    \item $S_{l, i}$: The exponent of bosonic annihilation operators acting on the $i^{th}$ bosonic mode within the $l^{th}$ term.
    \item $R_{l, i}$: The exponent of bosonic creation operators acting on the $i^{th}$ bosonic mode within the $l^{th}$ term.
    \item $P$: The sum of the exponents of bosonic ladder operators acting on all modes within a single term.
\end{itemize}