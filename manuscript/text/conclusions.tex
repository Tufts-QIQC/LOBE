\section{Conclusions}
\label{sec:conclusions}

\textbf{Future Directions}
The application of the bosonic coefficients is a potential area of improvement for more efficient implementations.
Primarily, if one could coherently compute $\cos^{-1}{\sqrt{\omega_i}}$ either in-place or out-of-place, it may be possible to use techniques such as hamming weight phasing to apply the appropriate coefficient without the need for multiplexing over all possible occupation states.
Alternatively, if the coefficient qubits begin outside of the encoded subspace (as opposed to inside), it may be possible to multiplex only over the $\Omega - R$ occupation states that will not be "zeroed-out". 
This would reduce the number of Toffolis by $R$ ($S$/$T$) for each bosonic ladder operator.
