\section{Conclusions}
\label{sec:conclusions}

Block-encodings of Hamiltonians are important components for quantum simulation algorithms.
Two frameworks for constructing block-encodings that have gained traction in the field are sparse block-encodings and LCU,with the latter being restricted to operators written as a linear combination of unitaries.

In this work, we presented a framework that bridges the gap between sparse and LCU block-encodings which we refer to as LOBE (Ladder Operator Block-Encoding).
LOBE can be used to construct block-encodings of operators written as a linear combination of products of ladder operators, providing a clear extension of LCU methods.

In this work, we provided schematics for constructing block-encodings using LOBE of Hamiltonians that act on fermionic, antifermionic, and bosonic modes.
We gave both analytical and numerical quantum resource estimates for the Yukawa model and the $\phi^4$ model using two different variants of LOBE.
Notably, we show that the number of qubits required scales logarithmically with both the number of terms in the Hamiltonian and the bosonic occupation cutoff.
Additionally, we show that the number of non-Clifford operations scales linearly with both the number of terms and the bosonic occupation cutoff.
Lastly, the rescaling factor of the block-encoding scales polynomially with both the number of terms and the bosonic occupation cutoff.

Hamiltonians of quantum systems that arise in high-energy physics and quantum field theory are often written as a linear combination of ladder operators acting on fermions, antifermions, and bosons.
LOBE introduces a clear method to construct efficient block-encodings for such systems, providing a key ingredient for composing quantum simluation algorithms of models arising in both high-energy physics and quantum field theory.