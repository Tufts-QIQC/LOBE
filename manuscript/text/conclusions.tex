\section{Conclusions}
\label{sec:conclusions}

In this work, we detail a framework which we refer to as LOBE (Ladder Operator Block-Encoding) which constructs block-encoding circuits for operators written in second-quantization.
We give explicit compilations for operators written as linear combinations of products of ladder operators acting on both fermionic and bosonic modes.
With this framework, many quantum operators of interest - such as Hamiltonians - can be block-encoded directly in their second-quantized form.
This avoids expanding operators in the Pauli basis prior to block-encoding which introduces a signficant overhead.

Additionally, we provide analytical and numerical costs for the relevant quantum resources required by LOBE for various operators of interest.
We compare our constructions to those that require the Jordan-Wigner and Standard Binary transformations to express ladder operators in the Pauli basis.
Our numerical results show that in most relevant cases, the LOBE framework produces block-encodings that require signficantly fewer T gates, non-Clifford rotations, block-encoding ancillae and total number of qubits and result in block-encodings with smaller rescaling factors.
In addition, the LOBE framework has better asymptotic scaling with respect to several parameters such as the maximum occupation of bosonic modes, the number of momentum modes, and the locality of the operator.
Notably, the LOBE constructions result in exponentially fewer non-Clifford operations for block-encoding products of ladder operators with respect to the locality as compared to a naive expansion of the operator in the Pauli basis.

