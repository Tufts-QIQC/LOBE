\section{LCU Encodings}
\label{sec:lcu-encodings}

The LCU results that have been given throughout this paper have done so by taking the ladder operator term, mapping it to Pauli operators in a particular manner, and then doing a LCU block encoding of the Pauli operators. 
The fermionic ladder operators are mapped under the \textit{Jordan-Wigner (JW)} \cite{jordan-wigner} transform, while the bosonic operators are mapped under the \textit{standard binary (SB)} \cite{Sawaya_2020, McArdle_2019}encoding.
The explicit mappings can be constructed as follows. 

\subsection{Jordan-Wigner (JW) Encoding}
\label{subsec:jordan-wigner}
The fermionic ladder operators increase or decrease the occupation number of the fermionic state by 1 for creation and annihilation operators respectively. 
The corresponding qubit operator that accomplishes this on a given mode (qubit) is given by $\frac12 \left(X \mp iY \right)_i$ for creation ($-$) and annihilation ($+$) operators on mode $i$. 
This is not the entire transformation; however, because it does not pick up the property parity exchange rules for fermions.
A string of Pauli $Z$'s are included on qubits less than the mode being acted on to satisfy this requirement.

The full mapping from ladder operator $\rightarrow$ Pauli operator is:

\begin{align}
    b_i &= \frac12 \left(X_i + iY_i\right) \otimes Z_{i - 1} \otimes \dots \otimes Z_0 \\ \nonumber
    b_i^\dagger &= \frac12 \left(X_i - iY_i\right) \otimes Z_{i - 1} \otimes \dots \otimes Z_0\\
\end{align}

Identities may be tensored on for any mode $j > i$ if necessary.


\subsection{Standard-Binary (SB) Encoding}
\label{subsec:standard-binary}

The standard binary encoding is chosen over another encoding (such as the unary mapping) because it respects the encoding system used in the LOBE circuits.
That is, each mode in the state is represented by $\lceil \log_2{\Omega} \rceil$ qubits in binary representing the current occupancy of bosons in the given mode. 
The mapping comes from the observation:

\begin{align}
    a_i^\dagger &= \sum_{s = 0}^{\Omega} \sqrt{s + 1}|s + 1\rangle \langle s|\\ \nonumber
    a_i &= \sum_{s = 0}^{\Omega} \sqrt{s + 1}|s\rangle \langle s + 1|.\\
\end{align}

A term in this sum like $|2\rangle \langle 3|$ can be written in binary as $|10\rangle \langle 11|$. 
We can compute this outer product by separating out the qubits as $|1\rangle \langle 1| \otimes |0\rangle \langle 1|$.
The following relevant outer products are :

\begin{align*}
    |0\rangle \langle 0| &= \frac12 (I + Z)\\
    |1\rangle \langle 1| &= \frac12 (I - Z)\\
    |0\rangle \langle 1| &= \frac12 (X + iY)\\
    |1\rangle \langle 0| &= \frac12 (X - iY)\\
\end{align*}
