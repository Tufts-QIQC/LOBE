\section{Constructing Ladder Operator Oracles}
\label{sec:ladder-op-oracles}

In this section, we discuss the explicit construction of the oracles used to apply different ladder operators onto the quantum state.
We separate these oracles based on the type of particle that they act on: fermionic ($O_b$), antifermionic ($O_d$), and bosonic ($O_a$).
Application of a product of ladder operators or a ``term'' as defined by Eq. \ref{eq:term} is achieved by successively applying the corresponding oracles for each operator included in the term.   

\subsection{Encoding}
\label{subsec:encoding}

Before discussing the compilation of these oracles, we first define the encoding of the system that we use in this work.
Different system encodings will require different oracle constructions for applying operators.

In this work, the encoding that is used for the occupation of the fermionic modes is identical to the Jordan-Wigner encoding \cite{jordan-wigner}.
The map between a Fock state to a qubit state is given as 
\begin{equation}
    \ket{n_{I_b}, \dots, n_{1_b}, n_{0_b}} \rightarrow \ket{q_{I_b}, \dots, q_{1_b}, q_{0_b}}
\end{equation}
where $n_{i_b} = q_{i_b} \in [0, 1]$ depending on if mode $i$ is occupied or not.
The encoding scheme is the same for antifermions.
In total, the number of qubits required for the fermionic and antifermionic susbsystems is: $Q_{\psi_b} = I_b$ and $Q_{\psi_d} = I_d$.

The encoding scheme for bosons must allow for occupancies in the range $[0, \Omega]$ due to the absence of the Pauli exclusion principle.
We choose to represent store the occupancy of a bosonic mode in binary notation: 
\begin{equation}
    \ket{n_{i_a}} \rightarrow \ket{b_j, b_{j-1}, ..., b_0}
\end{equation}
where $j$ runs from $0$ to $\lceil \log_2{\Omega} \rceil - 1$ and the values of $b_j$ are given by the binary representation of $n_{i_a}$.
This is identical to the encoding used in \cite{rhodes2024exponential}. \ws{check citation}

Storing the occupation of each bosonic mode requires $\lceil \log_2{\Omega} \rceil$ qubits if we assume that the maximum bosonic occupation is the same for each bosonic mode.
This choice of the maximum bosonic occupation being constant for all bosonic modes is not a restriction of the methods presented in this work, but is chosen for simplicity. 
With this choice, the number of qubits required for the bosonic subsystem is: $Q_{\psi_a} = I_a \lceil \log_2{\Omega} \rceil$.

\subsection{Fermionic Ladder Operators}

\subsubsection{Encoding}

\subsubsection{Action}

\ws{
    \begin{enumerate}
        \item Desired action
        \item Define action as non-unitary operator (flip coefficient qubit, high-level circuit)
        \item Applying phase pickup/coefficient
        \item Update Occupancy
    \end{enumerate}
}

In this section, we aim to define a family of oracles ($\{O_{b^\dagger_i}, O_{b_i}\}$) that apply the associated fermionic creation ($b_i^\dagger$) and annihilation ($b_i$) operators onto the quantum state.
As these are non-unitary operators, we define these oracles to be of the form of Eq. \ref{eq:general-block-encoding}:
\begin{equation}
    O_{b^\dagger_i} \ket{\psi} \ket{0}_\text{anc} = b^\dagger_i \ket{\psi}\ket{0}_\text{anc} + \ket{\text{junk}}\ket{0_\perp}_\text{anc}
\end{equation}

Since the fermionic (and antifermionic) ladder operators are already scaled such that their spectral norm is less than $1$, we do not need to introduce any rescaling factor for these operators.

We will begin by discussing the action of applying a fermionic creation operator on the $i^\text{th}$ mode: $b_i^\dagger$.
Recall the definition of this operator given by equation \ref{eq:fermionic-creation}.

We can first notice that the action of the operator creates two implications.
First, if the mode is already occupied, then state in our encoded subspace should be zeroed-out. 
Second, if the mode is unoccupied, then the mode should become occupied and a sign should be applied based on the occupancy of the preceeding fermionic modes.


The sign of the output state can be achieved simply by applying Pauli-Z operators to each of the fermionic modes with index $j < i$.
Each of these operators is a Clifford operator, even if controlled.


\subsection{Bosonic Ladder Operators}

\subsection{Encoding}

\ws{
\begin{enumerate}
    \item Chosen encoding
    \item Bosonic cutoffs
\end{enumerate}
}

\subsection{Action}

\ws{
    \begin{enumerate}
        \item Desired action
        \item Define rescaled operators
        \item Applying bosonic coefficients 
        \item Updating Occupancy
        \item Applying single bosonic op
        \item Applying multiple bosonic operators on same mode
        \item Applying multiple bosonic operators on different modes
    \end{enumerate}
}

\subsection{Terms}

\ws{
    How do we apply a term?
    \begin{enumerate}
        \item Define rescaled term and applying rescaled term with updated coefficient
        \item Applying term by sequentially applying ladder-op oracles 
    \end{enumerate}
}