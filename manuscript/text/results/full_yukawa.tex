\subsection{Full Yukawa}
\label{sec:yukawa_results}

Here, we examine the spacetime cost to block-encode the full Yukawa model which includes interactions between fermionic, antifermionic, and bosonic modes.  
This is a theory of interacting fermions and bosons which can be used as a model of the strong nuclear force between hadrons.
The Lagrangian for this model is given as:
\begin{equation}
    \label{eq:yukawa-lagrangian}
    \mathcal{L} = \bar \psi \left(i\gamma^\mu \partial_\mu - m \right)\psi + \frac{1}{2}\partial_\mu \phi \partial^\mu \phi - \frac{1}{2}\mu^2\phi^2 - g\bar \psi \psi \phi,
\end{equation}
where $m_f$ is the mass of the fermion, $m_b$ is the mass of the boson, and $g$ describes the strength of the interaction.
Unlike $\phi^4$ theory, the Yukawa model involves multiple types of fields including a fermionic (Dirac) field ($\psi$) and the conjugate (antifermionic) Dirac ($\bar \psi$) field, in addition to the bosonic field ($\phi$).

The Hamiltonian associated with this Lagrangian can be written in second-quantization as:
\begin{align}
    \begin{split}
        H = &\sum_i c_i b_i^\dagger b_i + \sum_i c_i d_i^\dagger d_i + \sum_i c_i a_i^\dagger a_i + \\
        &\sum_{ijk}c_{ijk}\left(b_i^\dagger b_j a_k^\dagger + h.c. \right) + \sum_{ijk}c_{ijk}\left(d_i^\dagger d_j a_k^\dagger + h.c. \right) + \\
        &\sum_{ijk}c_{ijk}\left(b_i^\dagger d_j^\dagger a_k + h.c. \right) + \sum_{ijkl}c_{ijkl}b_i^\dagger b_j a_k^\dagger a_l + \\
        &\sum_{ijkl}c_{ijkl}d_i^\dagger d_j a_k^\dagger a_l + \sum_{ijkl}c_{ijkl}\left(b_i^\dagger d_j^\dagger a_k a_l + h.c. \right)
    \end{split}
\end{align}
where the values of the coefficients can be determined by classical preprocessing.

\begin{figure*}
    \label{fig:full-yukawa}
    \includegraphics[width = 16cm]{figures/full-yukawa-resolution-3.pdf}
    \caption{
        \textbf{Full Yukawa}
        The number of T gates (upper-left), number of non-Clifford rotations (lower-left), block-encoding ancillae (upper-middle), maximum number of qubits used (lower-middle), and rescaling factor (lower-right) are shown as a function of the number of momentum modes.
        The bosonic cutoff is fixed to $\Omega = 3$ and the parameters $m_f$, $m_b$, and $g$ are set to $1$ for all data points.
        Results for the ``Pauli - Expansion" method are shown as the orange triangles and results for LOBE are shown as the blue circles.
    }
\end{figure*}

In Figure \ref{fig:full-yukawa}, the spacetime costs for both the Pauli Expansion and LOBE block-encodings are shown as a function of the number of momentum modes.
For this model, the LOBE constructions result in fewer required quantum resources for all metrics.
Notably, the number of T gates and number of non-Clifford rotations required for LOBE is approximately two orders of magnitude smaller than the Pauli Expansion construction when the number of modes is $12$.
