\subsection{Full Yukawa Results}
\label{sec:yukawa_results}

The first interacting field theory between different types of particles studied is generally the Yukawa model.
This is a theory of interacting fermions and bosons which can be used as a model of the strong nuclear force between hadrons.
The Lagrangian for this model is given as

\begin{equation}
    \label{eq:yukawa-lagrangian}
    \mathcal{L} = \bar \psi \left(i\gamma^\mu \partial_\mu - m \right)\psi + \frac{1}{2}\partial_\mu \phi \partial^\mu \phi - \frac{1}{2}\mu^2\phi^2 - g\bar \psi \psi \phi,
\end{equation}

The Hamiltonian can be written, again with coefficients obscured, as:

\begin{align}
    \begin{split}
        H = &\sum_i c_i b_i^\dagger b_i + \sum_i c_i d_i^\dagger d_i + \sum_i c_i a_i^\dagger a_i + \\
        &\sum_{ijk}c_{ijk}\left(b_i^\dagger b_j a_k^\dagger + h.c. \right) + \sum_{ijk}c_{ijk}\left(d_i^\dagger d_j a_k^\dagger + h.c. \right) + \\
        &\sum_{ijk}c_{ijk}\left(b_i^\dagger d_j^\dagger a_k + h.c. \right) + \sum_{ijkl}c_{ijkl}b_i^\dagger b_j a_k^\dagger a_l + \\
        &\sum_{ijkl}c_{ijkl}d_i^\dagger d_j a_k^\dagger a_l + \sum_{ijkl}c_{ijkl}\left(b_i^\dagger d_j^\dagger a_k a_l + h.c. \right)
    \end{split}
\end{align}

Unlike $\phi^4$ theory, the Yukawa model involves another type of field, a fermionic (Dirac) field $\psi$ describing fermionic particles.
$\bar \psi$ is the conjugate Dirac field, $m_f$ is the mass of the fermion, $m_b$ is the mass of the boson, and $g$ again describes the strength of the interaction.