\subsection{Quartic Harmonic Oscillator Results}
\label{sec:qosc_results}

The quartic harmonic oscillator \cite{PhysRev.184.1231, girguś2024spiralflowquantumquartic, wójcik2012applicationnumericalrenormalizationgroup} is an extension of the standard harmonic harmonic with an additional non-linearity of the form $\lambda \varphi^4$. \ws{@Guz, what does the $\lambda$ and $\varphi^4$ mean here? I'm assuming $\lambda$ is a coefficient? and $\varphi^4$ is a field or something? Can you write one short sentence to clarify this?}
This leads to the Hamiltonian $H = \frac12\dot\phi^2 + \frac{m}{2}\phi^2 + gm^3\phi^4 $, where $\phi$ represents a bosonic field composed of a single mode. \ws{Likewise, need to define the parameters ($m$ and $g$) and $\dot\phi$.}

In a second-quantized, dimensionless form, the Hamiltonian can be written as:
\begin{equation}
    \label{eq:qosc}
    H = a^\dagger a + g\left(a + a^\dagger \right)^4
\end{equation}
where there is only one bosonic mode.

After expanding the product, normal ordering all terms, and removing the constant offset, this Hamiltonian can be written as a linear combination of $8$ terms consisting of three pairs of operators plus their hermitian conjugates and two operators that are their own hermitian conjugates:
\begin{equation}
    \begin{split}
        H = &(12g + 1) a^\dagger a + 6g a^{\dagger^2} a^2 + 6g \left(a^{\dagger^2} + a^2 \right) \\
        + &4g \left(a^{\dagger^3} a + a^\dagger a^3 \right) + g \left(a^{\dagger^4} + a^4 \right)
    \end{split}
\end{equation}

\begin{figure*}
    \label{fig:qosc}
    \includegraphics[width = 16cm]{figures/quartic_oscillator.pdf}
    \caption{
        \textbf{Quartic Harmonic Oscillator}
        The number of T gates (upper-left), number of non-Clifford rotations (lower-left), block-encoding ancillae (upper-middle), maximum number of qubits used (lower-middle), and rescaling factor (lower-right) are shown as a function of the bosonic occupation cutoff ($\Omega$).
        The parameter $g$ is set to $1$ for all data points.
        Results for the Pauli (LCU) method are shown as the orange triangles and results for LOBE are shown as the blue circles.
        The optimal rescaling factor, which is given by the L2 norm of the Hamiltonian, is shown as the dashed black crosses.
    }
\end{figure*}

Since there is only a single active bosonic mode, both methods that expand the operators in the Pauli basis result in the same construction therefore we refer to these constructions collectively by ``Pauli (LCU)''.
In Figure \ref{fig:qosc}, we show the scaling of the spacetime costs associated with both the LOBE and Pauli (LCU) block-encodings as a function of the bosonic occupation cutoff ($\Omega$).

For small values of $\Omega$, the Pauli (LCU) block-encodings result in lower spacetime costs, however, the LOBE constructions have favorable scaling and therefore a crossover point is seen.
For the time-complexity, this crossover occurs at $\Omega = 15$ for the number of T gates and $\Omega = 7$ for the number of non-Clifford rotations.
For the space-complexity, this crossover occurs at $\Omega = 7$ for the number of block-encoding ancillae and $\Omega = 31$ for the maximum number of qubits.
Finally, for the rescaling factor, this crossover occurs at $\Omega = 31$.
