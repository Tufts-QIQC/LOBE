\subsection{$\phi^4$ Results}
\label{sec:phi4_results}

The first interesting relativistic Hamiltonian studied with LOBE is $\phi^4$ theory. 
Any field theory is defined at the level of a Lagrangian, which for $\phi^4$ theory is given as
\begin{equation}
    \mathcal{L} = \frac12 \left(\partial_\mu \phi \right)^2 - \frac{m^2}{2}\phi^2 - g\phi^4.
\end{equation}

To obtain a Hamiltonian from a Lagrangian, a Legendre transformation is performed, in which an explicit set of coordinates must be chosen. 
The set of coordinates used in this paper, which lead to the simplest forms of the corresponding Hamiltonians, are front form (lightfront) coordinates \cite{Dirac1949}.
A discussion on lightfront coordinates, as well as the corresponding Hamiltonians for each theory studied will be given in appendix \ref{subsec:lightfront-hamiltonian}.

Obscuring the coefficients, which can be found in appendix \ref{subsec:lightfront-hamiltonian}, the $\phi^4$ Hamiltonian can be written as:

\begin{align}
    H = \sum_i c_i a_i^\dagger a_i + \sum_{ijkl}c_{ijkl} \left(a_i^\dagger a_j^\dagger a_k^\dagger a_l + h.c. \right) + \sum_{ijkl}c_{ijkl}a_i^\dagger a_j^\dagger a_k a_l
\end{align}

Unlike the non-relativistic theories, both relativistic theories are defined over many modes.
Thus, the results will be given as two sets of plots: one set of cost vs. \textit{resolution} (a discussion on resolution is given in the appendix. For those uninterested, this can be loosely thought of as number of modes), and one set of cost vs occupancy ($\Omega$.)

\begin{figure*}
    \label{fig:phi4}
    \includegraphics[width = 16cm]{figures/phi4-resolution.pdf}
    \caption{
        \textbf{$\phi^4$}
        The number of T gates (upper-left), number of non-Clifford rotations (lower-left), block-encoding ancillae (upper-middle), maximum number of qubits used (lower-middle), and rescaling factor (lower-right) are shown as a function of the resolution ($K$).
        The bosonic cutoff is fixed to $\Omega = 3$ and the parameters $g$ and $m_b$ are set to $1$ for all data points.
        Results for the Pauli (LCU) method are shown as the orange triangles and results for LOBE are shown as the blue circles.
        The optimal rescaling factor, which is given by the L2 norm of the Hamiltonian, is shown as the dashed black crosses.
    }
\end{figure*}

% \begin{figure}[h]
%     \centering
%     \includegraphics[width = 15cm]{figures/phi4_occupancies.png}
%     \caption{}
%     \label{}
% \end{figure}