\section{Theory}

Give background of ladder operators and constructions of realistic Hamiltonians/Observables from ladder operators

\subsection{Encoding}
\label{subsec:encoding}

Here we'll discuss how we encode the physical states we are interested in terms of qubits/registers.

\subsection{Ladder Operators}
\label{subsec:operators}

\ws{@Gus, you probably have much better language to define all of this stuff. I just needed to write something down so I could reference it in the circuit construction. Don't hesitate to scrap anything in here.}


\subsubsection{Feromons and Antiferomons}

Define action of fermionic ladder operators.

Fermions (and antifermions) obey the Pauli-exclusion principle \ws{citation} and therefore the occupation of a (anti)fermionic mode can only be occupied ($\ket{1}$) or unoccupied ($\ket{0}$).
Fermionic (and antifermionic) ladder operators only act non-trivially on the qubits encoding the mode that the ladder operator acts on and we define their action as follows.

The fermionic creation operator is given by:
\begin{equation}
    b_i^\dagger \ket{n_{b_i}} = 
    \begin{cases} 
        (-1)^{\sum_{j < i} b_j} \ket{1}  & when \ket{n_{b_i}} is \ket{0} \\
        0 & when \ket{n_{b_i}} is \ket{1}
    \end{cases}
\end{equation}
where $b_i$ denotes a fermionic ladder operator on the $i^{th}$ mode, the $^\dagger$ indicates a creation operator, and $\ket{n_{b_i}}$ is the occupation of the $i^{th}$ fermionic mode.
An antifermionic creation operator is defined as above with the symbol $d$ to denote that the operator acts on antifermions.

For a fermionic creation operator, if the mode being acted upon is unoccupied, then the creation operator "creates" a fermion in that mode and applies a phase determined by the parity of the occupation of the previous modes.
Therefore the ordering of the modes in the encoding has an implication on the action of the operator that must be accounted for.
Since fermionic modes can only be either occupied or unoccupied, then if the mode is already occupied the operator zeroes the amplitude of the quantum state, thereby "destoying" that portion of the quantum state. 

The fermionic annihilation operator is given by:
\begin{equation}
    b_i \ket{n_{b_i}} = 
    \begin{cases} 
        (-1)^{\sum_{j < i} b_j} \ket{0}  & when \ket{n_{b_i}} = \ket{1} \\
        0 & when \ket{n_{b_i}} = \ket{0}
    \end{cases}
\end{equation}
and the antifermionic annihilation operator is likewise defined for $d$ instead of $b$.

The action of the annihilation operators is similar (and opposite) to the creation operators.
If the mode is already occupied, then the annihilation operator "annihilates" the fermion at that mode by setting the occupation to zero and applies a phase based on the parity of the occupation of the preceeding modes.
If the mode is unoccupied before the operator is applied, then the annihilation operator zeroes the amplitude.

\subsubsection{Bosos}

\begin{equation}
    a_i^\dagger \ket{n_{a_i}} = 
    \begin{cases} 
        \sqrt{n_{a_i} + 1} \ket{n_{a_i} + 1}  & when \ket{n_{a_i}} \neq \ket{\Omega} \\
        0 & when \ket{n_{a_i}} = \ket{\Omega}
    \end{cases}
\end{equation}
where $a_i$ denotes a bosonic ladder operator on the $i^{th}$ mode, the $^\dagger$ indicates a creation operator, $\ket{n_{a_i}}$ is the occupation of the $i^{th}$ bosonic mode, and $\Omega$ is the maximum allowable bosonic occupation.

\begin{equation}
    a_i \ket{n_{a_i}} = 
    \begin{cases} 
        \sqrt{n_{a_i}} \ket{n_{a_i} - 1}  & when \ket{n_{a_i}} \neq \ket{0} \\
        0 & when \ket{n_{a_i}} = \ket{0}
    \end{cases}
\end{equation}

\subsubsection{Commutation Rules}
\label{subsec:commutation}


\subsection{Observables}
\label{subsec:observables}

\subsubsection{Products of Ladder Operators (Terms)}

We define a \textit{term} ($T$) as a product of ladder operators that can act on fermionic, antifermionic, and bosonic modes:
\begin{equation}
    T = \prod_{m=0}^{M-1} c_m
\end{equation}
where $M$ is the number of ladder operators in the term and $c_m \in \{b_i, b_i^\dagger, d_i, d_i^\dagger, a_i, a_i^\dagger\}$.

The ladder operators ($c_m$) can be reordered arbitrarily with the introduction of additional terms due to the commutation rules described in \ref{subsec:commutation}.
In this work, we will have a preference for \textit{normal ordering} of the operators to obey the following structure:
\begin{equation}
    T = \Big( \prod_i (\delta_{b_i^\dagger} b_i^\dagger)(\delta_{b_i} b_i) \Big) \Big( \prod_i (\delta_{d_i^\dagger} d_i^\dagger)(\delta_{d_i} d_i) \Big) \Big( \prod_i (\delta_{a_i^\dagger} (a_i^\dagger)^r)(\delta_{a_i} (a_i)^s) \Big) 
\end{equation}
where $\delta$ takes the value $0$ or $1$ to denote if the operator is active in the term and the values $r$ and $s$ are positive integers $\in [0, \Omega]$ and denote the exponent of the bosonic ladder operators acting on that particular bosonic mode.

\ws{Describe number/occupation operators acting on fermions/antifermions and bosons and rewrite previous equation for $T$ including these operators.}

\subsubsection{Linear Combinations of Terms}

We can write Hamiltonians (or observables) in the form of linear combinations of terms:
\begin{equation}
    \label{eq:lclo}
    H = \sum_{l=0}^{L-1} \alpha_l T_l
\end{equation}
where $L$ is the total number of terms and $\alpha_l$ is a real-valued coefficient associated with the term $T_l$.