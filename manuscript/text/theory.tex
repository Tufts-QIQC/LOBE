\section{Theory}

%Give background of ladder operators and constructions of realistic Hamiltonians/Observables from ladder Operators
In quantum field theories and quantum chemistry, the main method of keeping track of multiparticle states is known as second quantization \cite{Sakurai_Napolitano_2020}.
In second quantization, multiparticle state vectors are written as $\ket{n} = \ket{n_\Lambda, \dots, n_1, n_0}$ where $n_i \in \mathbb{Z}$ is the number of particles present in mode $i$.
For fermions, $n_i \in [0, 1]$ necessarily due to the Pauli exclusion principle \gus{cite PEP}; whereas for bosons, $n_i \in [0, 1, 2, \dots)$.
The vector space of all second-quantized state vectors is called the Fock space, with the state vectors called Fock states. In second quantization, the Hilbert space is promoted to the Fock space via \cite{Schwartz_2013}:

\begin{equation}
    \mathcal{F} = \oplus_n \mathcal{H}_n.
\end{equation}

Second quantization apperas in quantum chemistry after projecting the Hamiltonian onto basis wavefunctions, and assuring exchange symmetry via Slater determinants. \gus{Will, you know more about quantum chemistry than me, so please edit this as needed}.
Similarly, in quantum field theories, field operators, rather than wavefunctions, are the main objects of the theory. These field operators act on second-quantized states to create and annihilate particles in the field. 

\subsection{Ladder Operators}
\label{subsec:operators}
In addition to Fock states, the other main aspect of second quantization is the presence of creation and annihilation operators, which collectively will be refered to as \emph{ladder operators}. These operators are essential parts of the theory that manipulate the Fock states and change the nature of the multiparticle system. 


%\ws{@Gus, you probably have much better language to define all of this stuff. I just needed to write something down so I could reference it in the circuit construction. Don't hesitate to scrap anything in here.}
\subsubsection{Feromons and Antiferomons}

%Define action of fermionic ladder operators.

Fermions (and antifermions) obey the Pauli-exclusion principle \ws{citation} and therefore the occupation of a (anti)fermionic mode can only be occupied ($\ket{1}$) or unoccupied ($\ket{0}$).
Fermionic (and antifermionic) ladder operators only act non-trivially on the qubits encoding the mode that the ladder operator acts on and we define their action as follows.

The fermionic creation operator is given by:
\begin{equation}
    b_i^\dagger \ket{n_{b_i}} = 
    \begin{cases} 
        (-1)^{\sum_{j < i} b_j} \ket{1}  & when \ket{n_{b_i}} is \ket{0} \\
        0 & when \ket{n_{b_i}} is \ket{1}
    \end{cases}
\end{equation}
where $b_i$ denotes a fermionic ladder operator on the $i^{th}$ mode, the $^\dagger$ indicates a creation operator, and $\ket{n_{b_i}}$ is the occupation of the $i^{th}$ fermionic mode.
An antifermionic creation operator is defined as above with the symbol $d$ to denote that the operator acts on antifermions.

For a fermionic creation operator, if the mode being acted upon is unoccupied, then the creation operator "creates" a fermion in that mode and applies a phase determined by the parity of the occupation of the previous modes.
Therefore the ordering of the modes in the encoding has an implication on the action of the operator that must be accounted for.
Since fermionic modes can only be either occupied or unoccupied, then if the mode is already occupied the operator zeroes the amplitude of the quantum state, thereby "destoying" that portion of the quantum state. 

The fermionic annihilation operator is given by:
\begin{equation}
    b_i \ket{n_{b_i}} = 
    \begin{cases} 
        (-1)^{\sum_{j < i} b_j} \ket{0}  & when \ket{n_{b_i}} = \ket{1} \\
        0 & when \ket{n_{b_i}} = \ket{0}
    \end{cases}
\end{equation}
and the antifermionic annihilation operator is likewise defined for $d$ instead of $b$.

The action of the annihilation operators is similar (and opposite) to the creation operators.
If the mode is already occupied, then the annihilation operator "annihilates" the fermion at that mode by setting the occupation to zero and applies a phase based on the parity of the occupation of the preceeding modes.
If the mode is unoccupied before the operator is applied, then the annihilation operator zeroes the amplitude.

\subsubsection{Bosos}

\begin{equation}
    \label{eq:bosonic-creation}
    a_i^\dagger \ket{n_{a_i}} = 
    \begin{cases} 
        \sqrt{n_{a_i} + 1} \ket{n_{a_i} + 1}  & when \ket{n_{a_i}} \neq \ket{\Omega} \\
        0 & when \ket{n_{a_i}} = \ket{\Omega}
    \end{cases}
\end{equation}
where $a_i$ denotes a bosonic ladder operator on the $i^{th}$ mode, the $^\dagger$ indicates a creation operator, $\ket{n_{a_i}}$ is the occupation of the $i^{th}$ bosonic mode, and $\Omega$ is the maximum allowable bosonic occupation.

\begin{equation}
    \label{eq:bosonic-annihilation}
    a_i \ket{n_{a_i}} = 
    \begin{cases} 
        \sqrt{n_{a_i}} \ket{n_{a_i} - 1}  & when \ket{n_{a_i}} \neq \ket{0} \\
        0 & when \ket{n_{a_i}} = \ket{0}
    \end{cases}
\end{equation}

\subsubsection{Hamiltonians via Fields}
Field operators in quantum field theories are written in terms of ladder operators as
\begin{equation}
    \phi(x) = \int \frac{d^3p}{(2\pi)^3}\frac{1}{\sqrt{2E_p}}\left(a_p e^{ipx} + a_p^\dagger e^{-ipx}\right).
\end{equation}
In classical field theory, the Hamiltonian is derived from the field operators via 

\begin{equation}
    H = \int d^3x \left(\frac{\partial \mathcal{L}}{\partial \dot{\phi}}\dot{\phi} - \mathcal{L} \right)
\end{equation}
\gus{Maybe this is unnecessary.}
This will necessarily lead to Hamiltonians written in terms of creation and annihilation operators, similarly to quantum chemistry. The main difference is that 
Hamiltonians in quantum field theory will contain products of \emph{different types} of particles. The block encoding presented in this paper takes this into account and allows any general quantum field theory Hamiltonian to be block encoded. 

\subsection{Commutation Rules, Normal and Canonical Ordering}
\label{subsec:commutation}
It is important to mention the commutation relations between the different types of ladder operators. They are given as:

\begin{align*}
    &\{b_i, b_j^\dagger\} = \{d_i, d_j^\dagger\} = [a_i, a_j^\dagger] = \delta_{ij}\\
    & [b_i, b_j] = [b_i^\dagger, b_j^\dagger] = 0 \\
    & [d_i, d_j] = [d_i^\dagger, d_j^\dagger] = 0 \\
    & [a_i, a_j] = [a_i^\dagger, a_j^\dagger] = 0 \\
    & \{b_i, d_j\} = \{b_i^\dagger, d_j\} = \{b_i, d_j^\dagger\} = \{b_i^\dagger, d_j^\dagger\} = 0\\
    & [f(a_i), f(b_i, d_j)] = 0
\end{align*}

LOBE requires each product of ladder operators to be \emph{canonically ordered}. Canonical ordering in quantum field theory is an adaptation of normal ordering. Given a product of ladder operators, the normal ordered version of the operator is that in which all creation operators are to the left of annihilation operators.
For example, the normal-ordered version of the operator $a_3 b_4^\dagger$ is $b_4^\dagger a_3$. We define a canonically ordered operator as an operator which is normal ordered among the individual particle types, but not necessarily overall. An example of this is $a_1^\dagger b_2 b_1^\dagger$ which canonically-ordered is $b_1^\dagger b_2 a_1^\dagger$.

When canonically (or normal) ordering a product of ladder operators, one cannot just push all creation operators to the left and annihilation operators to the right, as this will change the nature of the operator in general. When ordering a product, the commutation rules must be taken into account to ensure the operator after normal ordering is the same operator.
For example, given $b_2 b_1 b_1^\dagger$, the normal ordered version of this operator is $b_2 \left(\mathds{1} - b_1^\dagger b_1 \right) = b_2 + b_1^\dagger b_1 b_2$.

\subsection{Encoding}
\label{subsec:encoding}
In order to map information about the multiparticle system to Fock states (and thus qubit states), an encoding must be decided. Due to the different statistics obeyed by fermions and bosons, these encodings will be different for these different types of particles. 

The fermionic encoding used is the Jordan-Wigner encoding \cite{jordan-wigner}. The map between a Fock state to a qubit state is given as 
\begin{equation}
    \ket{n_\Lambda, \dots, n_1, n_0} \rightarrow \ket{q_\Lambda, \dots, q_1, q_0}
\end{equation}
where $n_i = q_i \in [0, 1]$ depending on if mode $i$ is occupied or not. Thus, mapping a state to a qubit register corresponds to intializing qubits in the $\ket{0}$ or $\ket{1}$ state for each wire in the register. 

Alternatively, for bosons, the encoding is slightly more complicated due to the absence of the Pauli exclusion principle. For bosons, we map a Fock state to a qubit register with $n_{\text{qubits}} = \Lambda \lceil \log_2{\Omega}\rceil$.
This is because for each mode $\lambda_i \in [0, \Lambda]$, we must encode the corresponding occupancy of bosons in this mode in binary. 

Thus, for a general Fock state with fermions, antifermions, and bosons, the total number of qubits needed to map this state to a qubit register is $\Lambda_{\text{fermi}} + \Lambda_{\text{antifermi}} + \Lambda_{\text{bose}} \lceil \log_2{\Omega}\rceil$.
Throughout this work, we assume that the mode cutoffs are the same for each type of particle $\Lambda_{\text{fermi}} + \Lambda_{\text{antifermi}} + \Lambda_{\text{bose}} \equiv \Lambda$.

%Here we'll discuss how we encode the physical states we are interested in terms of qubits/registers.

\ws{Question: do we want to also cost-out the unary encoding or just the "compact" encoding that we've been primarily working with?}




\subsection{Observables}
\label{subsec:observables}

\subsubsection{Products of Ladder Operators (Terms)}

We define a \textit{term} ($T$) as a product of ladder operators that can act on fermionic, antifermionic, and bosonic modes:
\begin{equation}
    T = \prod_{m=0}^{M-1} c_m
\end{equation}
where $M$ is the number of ladder operators in the term and $c_m \in \{b_i, b_i^\dagger, d_i, d_i^\dagger, a_i, a_i^\dagger\}$.

The ladder operators ($c_m$) can be reordered arbitrarily with the introduction of additional terms due to the commutation rules described in \ref{subsec:commutation}.
In this work, we will have a preference for \textit{normal ordering} \gus{\textit{canonically ordering}} the operators to obey the following structure:
\begin{equation}
    T = \Big( \prod_i ((a_i^\dagger)^S)(A_i)^R((a_i)^P) \Big) \Big( \prod_i \big( (d_i^\dagger)^{\delta_{d_i}^{\dagger}} (D_i)^{\delta_{D_i}} (d_i)^{\delta_{d_i}} \big) \Big) \Big( \prod_i \big( (b_i^\dagger)^{\delta_{b_i}^{\dagger}} (B_i)^{\delta_{B_i}} (b_i)^{\delta_{b_i}} \big) \Big) 
\end{equation}
\gus{Can we order this as fermions, antifermions, bosons, rather than the other way around? That is the style we have been using (they commute so it's technically the same).}
where $\delta$ takes the value $0$ or $1$ to denote if the operator is active in the term and the values $P$, $R$, and $S$ are integers $\in [0, \Omega]$ and denote the exponent of the bosonic ladder operators acting on that particular bosonic mode.
The operators $A_i$, $D_i$, and $B_i$ denote occupation operators acting on the $i^{th}$ bosonic, antifermionic, and fermionic modes respectively.

\ws{Describe number/occupation operators acting on fermions/antifermions and bosons and rewrite previous equation for $T$ including these operators.}

\subsubsection{Linear Combinations of Terms}

We can write Hamiltonians (or observables) in the form of linear combinations of terms:
\begin{equation}
    \label{eq:lclo}
    H = \sum_{l=0}^{L-1} \alpha_l T_l
\end{equation}
where $L$ is the total number of terms and $\alpha_l$ is a real-valued coefficient associated with the term $T_l$.