\section{Results}
\label{sec:results}

In this section, we demonstrate the use of LOBE to block-encode physically-relevant Hamiltonians: the Yukawa and $\phi^4$ model used in \ws{insert-blah here @gus @kamil}.
We study both variants of the block-encoding described in Sections \ref{sec:block-encoding} and expanded upon in Section \ref{sec:lobe} and give numerical counts for the numbers of non-Clifford operations, the number of ancillae required beyond the system register, and the rescaling factor imposed on the resulting block-encoding.

\subsection{Yukawa Theory}
The most basic quantum field theory that describes interactions between bosons and fermions is Yukawa theory \cite{Schwartz_2013}. The interaction term is $\mathcal{L}_{\text{Yukawa}} = \bar\psi \psi \phi$, which leads to interaction vertices containing fermions and/or antifermions and bosons.

An auspicious coordinate system can be utilized called the lightfront frame \cite{Dirac1949}. In this frame, it is straightforward to write down the Hamiltonian bound state eigenvalue equation in terms of creation and annihilation operators, without the ambiguous $\sqrt{\nabla^2 + m^2}$ term that appears in a canonical instant-time frame Hamiltonian approach. 

The interaction terms in the Hamiltonian are $H_{\text{Yukawa int}} \in \{b^\dagger b a, b^\dagger b a^\dagger, d^\dagger d a, d^\dagger d a^\dagger, b^\dagger d^\dagger a, bda^\dagger, b^\dagger b a^\dagger a, d^\dagger d a^\dagger a \}$, while the free part of the Hamiltonian is $H_{\text{free}} \in \{b^\dagger b, d^\dagger d, a^\dagger a \}$.
This Hamiltonian can be written out as an integral (discretized to a sum in order to block encode with LOBE) over these free and interaction ladder operators modulated with coefficients (which are not added here). 


\begin{figure}
    \centering
    \includegraphics[width=16cm]{figures/Yukawa_hamiltonian_gates_vs_terms.pdf}
    \caption{
        \textbf{Numerical Gate Counts for Increasing $I$ (Yukawa Hamiltonian).}
        The number of rotations (left), left-elbows (middle), and right-elbows (right) are plotted as a function of the number of terms in the Hamiltonian ($L$) for an increasing number of momentum modes ($I$).
        The gate counts for the variant of LOBE using \textit{USP} are shown as the blue squares.
        The gate counts for the variant of LOBE using \textit{ASP} are shown as the orange circles.
        The bosonic occupancy cutoff ($\Omega$) is set to $3$.
        The number of rotations excludes rotations by angles that result in Clifford operations.
    }
    \label{fig:Yukawa_hamiltonian_gates_vs_terms}
\end{figure}
\begin{figure}
    \centering
    \includegraphics[width=12cm]{figures/Yukawa_hamiltonian_qubits_and_rescaling_vs_terms.pdf}
    \caption{
        \textbf{Numerical Ancillae Counts and Rescaling Factors for Increasing $I$ (Yukawa Hamiltonian).}
        The number of required ancillae (left) and the resulting rescaling factor (right) for LOBE are plotted as a function of the number of terms in the Hamiltonian ($L$) for an increasing number of momentum modes ($I$).
        The counts for the variant of LOBE using \textit{USP} are shown as the blue squares.
        The counts for the variant of LOBE using \textit{ASP} are shown as the orange circles.
        The bosonic occupancy cutoff ($\Omega$) is set to $3$.
    }
    \label{fig:Yukawa_hamiltonian_qubits_and_rescaling_vs_terms}
\end{figure}

In Figure \ref{fig:Yukawa_hamiltonian_gates_vs_terms}, we plot the numerical gate counts estimates for the Yukawa Hamiltonian (Eq. \ref{eq:Yukawa-hamiltonian}) \gus{@will do we want to explicitly show the discretized Hamiltonian for both Yukawa and $\phi^4$ in the appendix?} as the number of terms in the Hamiltonain ($L$) increases.
The number of terms ($L$) increases directly with an increasing number of momentum modes and is given by $L = 1.3I^3 + 0.65I^2 - 0.4I $ \ws{confirm this, im just eyeballing rn} \gus{I did some fits to $L$ vs. $I$, and got this from np.polyfit (see Yukawa.ipynb)}.
For the three non-Clifford operations, the number of operations increases linearly with the number of terms in the Hamiltonian for this model.
Additionally, the two different compilation strategies (\textit{USP} (blue), \textit{ASP} orange) demonstrate the same numerical scaling and have nearly identical gate counts for all types of operations.
When the number of terms ($L$) is far from the next largest power of 2, \textit{ASP} requires more arbitrary rotations due to the compilation of Grover-Rudolph that is used in this work.

In Figure \ref{fig:Yukawa_hamiltonian_qubits_and_rescaling_vs_terms}, we plot the numerical estimates for the number of required ancillae (left) and the imposed rescaling factor (right) as the number of terms in the Hamiltonain ($L$) increases.
The number of ancillae grows logarithmically with the number of terms for both implementations.
The number of ancillae used for the index register grows logarithmically with the number of terms which accounts for this scaling.
The main advantage of the \textit{ASP} variant is the effect on the rescaling factor of the block-encoding.
While the rescaling factor of both variants seemingly grows linearly with respect to the number of terms in the Hamiltonian, the rescaling factor for the \textit{ASP} variant is significantly smaller than the \textit{USP} variant.
When block-encodings are employed as a subroutine in larger algorithms, the quantum resources for the algorithm are often dependent on the rescaling factor (with a lower rescaling factor generally being preferred).
For example, in the context of using Quantum Phase Estimation to estimate the eigenvalues of a Hamiltonian, the number of gates required typically scales as $O(\frac{1}{\lambda})$ \cite{babbush2018encoding}. 
However, the exact cost associated with the rescaling factor is difficult to determine without choosing a specific algorithm to benchmark with.

\begin{figure}
    \centering
    \includegraphics[width=16cm]{figures/Yukawa_hamiltonian_gates_vs_omega.pdf}
    \caption{
        \textbf{Numerical Gate Counts for Increasing $\Omega$ (Yukawa Hamiltonian) .}
        The number of rotations (left), left-elbows (middle), and right-elbows (right) are plotted as a function of the bosonic occupation cutoff ($\Omega$).
        The gate counts for the variant of LOBE using \textit{USP} are shown as the blue squares.
        The gate counts for the variant of LOBE using \textit{ASP} are shown as the orange circles.
        The number of momentum modes ($I$) is set to $2$.
        The number of rotations excludes rotations by angles that result in Clifford operations.
    }
    \label{fig:Yukawa_hamiltonian_gates_vs_omega}
\end{figure}
\begin{figure}
    \centering
    \includegraphics[width=12cm]{figures/Yukawa_hamiltonian_qubits_and_rescaling_vs_omega.pdf}
    \caption{
        \textbf{Numerical Ancillae Counts and Rescaling Factors for Increasing $\Omega$ (Yukawa Hamiltonian).}
        The number of required ancillae (left) and the resulting rescaling factor (right) for LOBE are plotted as a function of the bosonic occupation cutoff ($\Omega$).
        The counts for the variant of LOBE using \textit{USP} are shown as the blue squares.
        The counts for the variant of LOBE using \textit{ASP} are shown as the orange circles.
        The number of momentum modes ($I$) is set to $2$.
    }
    \label{fig:Yukawa_hamiltonian_qubits_and_rescaling_vs_omega}
\end{figure}

In Figure \ref{fig:Yukawa_hamiltonian_gates_vs_omega}, we plot the numerical gate counts estimates for the Yukawa Hamiltonian (Eq. \ref{eq:Yukawa-hamiltonian}) as the cutoff on the maximum bosonic occupation ($\Omega$) increases.
For the three non-Clifford operations, the number of operations increases linearly with the bosonic occupancy cutoff for this model.
Similar to the case with increasing $I$, the two different compilation strategies (\textit{USP} (blue), \textit{ASP} orange) demonstrate the same numerical scaling and have nearly identical gate counts for all types of operations.

In Figure \ref{fig:Yukawa_hamiltonian_qubits_and_rescaling_vs_omega}, we plot the numerical estimates for the number of required ancillae (left) and the imposed rescaling factor (right) as the cutoff on the maximum bosonic occupation ($\Omega$) increases.
The number of ancillae grows logarithmically with the bosonic occupation cutoff for both implementations.
The number of ancillae needed to update the bosonic occupancy grows logarithmically with $\Omega$ (Eq. \ref{eq:ancillae-bosonic-updates}) which accounts for this scaling.
Again, the main advantage of the \textit{ASP} variant is that the imposed rescaling factor is significantly smaller than the \textit{USP} variant, especially for large values of $\Omega$ despite the asymptotic scaling being linear for both.


\subsection{$\phi^4$ Theory}
\gus{@kamil}