\section{Renormalization Group Procedure for Effective Particles (RGPEP)}
\label{sec:rgpep}

Divergences appering in Hamiltonian quantum field theories come from far off-diagonal matrix elements. 
These matrix elements arise from interactions between particles of vastly different energy scales. \cite{}
The similarity renormalization group \cite{} applies a unitary transformation $U(\lambda)$ to a matrix $H$, where $\lambda$ is interpretted as an energy scale. 
In general, the unitary transformation defined by $H(\lambda) = U(\lambda)HU^\dagger(\lambda)$ ``lowers the energy scale'' in which interactions can happen. 
This is realized by a continuous transformation to $H$, parameterized by $\lambda$ that makes the original matrix increasingly diagonal. 
In doing this, the spectrum is preserved, but the matrix elements that lead to divergences are removed.

In practice, the parameter $\lambda$ is exchanged for $t$, where $t \sim 1/\lambda$. 
This is because it is easier to scale $t$ from $0 \rightarrow \infty$, than to scale $\lambda$ from $\infty \rightarrow 0$ (higher energy scales to lower).
Determining the rate of change in $t$ of the Hamiltonian $H$ leads us to determining a form of $U(t)$:

\begin{equation}
    \label{eq:rgpep-ut}
    \frac{dH(t)}{dt} = \left[\frac{dU(t)}{dt}U^\dagger(t), H(t) \right]
\end{equation}

The first term in the commutator, $\frac{dU(t)}{dt}U^\dagger(t)$ is called the \textit{generator} $\left(\mathcal{G}(t) \right)$, which is what is chosen, rather than $U(t)$ itself.
In general, any generator can be chosen (in the same way that any $U$ could be); however, the generator that ``pushes'' the matrix towards the diagonal, and is commonly used in literature is $\mathcal{G}(t) \equiv \left[H_0, H(t) \right]$
Thus, the RGPEP equation is given as 

\begin{equation}
    \label{eq:rgpep}
    \frac{dH(t)}{dt} = \left[\mathcal{G}(t), H(t) \right]
\end{equation}

This equation holds true for any matrix; however, in quantum field theories, the Hamiltonians are written as integrals of creation and annihilation operators. 
Given a set of Fock states, one could construct a matrix, and solve equation \ref{eq:rgpep}; however, the renormalization group procedure for effective particles (RGPEP) \cite{} offers an alternative approach.

RGPEP is based on the similarity renormalization group, but rather than calculating commutators of matrices, it is done by commutators of creation and annihilation operators. 
In general, the RGPEP equation cannot be solved exactly, so it must be solved order-by-order in $g$.
This is accomplished by expanding $H$ as 

\begin{equation}
    \label{eq:H-expansion}
    H(t) = H_0(t) + gH^{(1)}(t) + g^2 H^{(2)}(t) + \mathcal{O}(g^3).
\end{equation}

Asserting this form of $H(t)$ in equation \ref{eq:rgpep}, we obtain

\begin{equation}
    g\dot{H}^{(1)}(t) + g^2\dot{H}^{(2)}(t) + \dots = \left[\left[H_0, H_0 + gH^{(1)}(t) + g^2H^{(2)}(t) + \dots\right], H_0 + gH^{(1)}(t) + g^2H^{(2)}(t) + \dots\right].
\end{equation}

Thus, 

\begin{align}
    \label{eq:rgpep-order-by-order}
    &\mathcal{O}(g^0): \dot{H}_0 = 0\\ \nonumber
    &\mathcal{O}(g^1): g\dot{H}^{(1)}(t) = \left[\left[H_0, gH^{(1)}(t)\right], H_0\right] \\\nonumber
    &\mathcal{O}(g^2): g^2\dot{H}^{(2)}(t) = \left[\left[H_0, g^2H^{(2)}(t)\right], H_0\right] + \left[\left[H_0, gH^{(1)}(t)\right],gH^{(1)}(t)\right] \\ \nonumber
    &\vdots
\end{align}

The first equation is trivial, with solution $H_0(t) = H_0$. 
Higher order solutions, however, are not as simple to solve.
Appendix \ref{sec:first-order} describes the procedure to solve equations \ref{eq:rgpep-order-by-order} up to $\mathcal{O}(g^2)$, and gives the explicit forms of $H^{(1)}(t)$ and $H^{(2)}(t)$.

