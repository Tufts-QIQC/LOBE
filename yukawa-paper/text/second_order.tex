\subsubsection{$\mathcal{O}(g^2)$ Solution}
\label{sec:second-order}

At second order in $g$, the RGPEP equation is

\begin{equation}
    \label{eq:second-order}
    \frac{dH^{(2)}(t)}{dt} = \left[\left[H_0, H^{(2)}(t)\right], H_0\right] + \left[\left[H_0, H^{(1)}(t)\right],H^{(1)}(t)\right].
\end{equation}
This equation is more involved than the first order equation; however, looking at the parts of the sum on the right-hand side individually is insightful.

First, look at $\left[\mathcal{G}^{(1)}(t),H^{(1)}(t)\right],$ where the substitution $\mathcal{G}^{(1)}(t) = \left[H_0, H^{(1)}(t)\right]$ was made.

\begin{align*}
    \left[\mathcal{G}^{(1)}(t),H^{(1)}(t)\right] &= \left(-\mathcal{Q}^- H^{(1)}(t)\right)H^{(1)}(t) - H^{(1)}(t)\left(-\mathcal{Q}^- H^{(1)}(t)\right)\\
    &= :\left(-\mathcal{Q}^- H^{(1)}(t)\right)H^{(1)}(t): + :C\left[\left(-\mathcal{Q}^- H^{(1)}(t)\right)H^{(1)}(t)\right]:\\
    & - :H^{(1)}(t)\left(-\mathcal{Q}^- H^{(1)}(t)\right): - :C\left[H^{(1)}(t)\left(-\mathcal{Q}^- H^{(1)}(t)\right)\right]:\\
    &= :C\left[\left(-\mathcal{Q}^- H^{(1)}(t)\right)H^{(1)}(t)\right]: - :C\left[H^{(1)}(t)\left(-\mathcal{Q}^- H^{(1)}(t)\right)\right]:.
\end{align*}

The last line of this equality is true because there are an equal number of fermions and antifermions, thus their normal orderings cancel.
This difference in contractions can be simplified to the following expression:

\begin{equation}
    \label{eq:GH-comm}
    \left[\mathcal{G}^{(1)}(t),H^{(1)}(t)\right] = \int\limits_{\substack{q_1, q_2, q_3 \\ q_1', q_2', q_3'}} \delta \left(q_1^+ + q_2^+ + q_3^+ \right)\delta \left(q_1'^+ + q_2'^+ + q_3'^+ \right)A(t):C\left[H^{(1)}(t)H^{'(1)}(t) \right]:
\end{equation}
where $$A(t) = \left[-\left(q_1^+ + q_2^+ + q_3^+ \right) + \left(q_1'^+ + q_2'^+ + q_3'^+ \right)\right]f(t) f'(t).$$
A pictorial proof of equation \ref{eq:GH-comm} will be given in the appendix.

To determine the first commutator in equation \ref{eq:second-order}, $H^{(2)}(t)$ must be parameterized similar to the $\mathcal{O}(g)$ solution. 
$H^{(2)}(t)$ will not only contain the $\mathcal{O}(g^2)$ instantaneous terms that exist in the canonical Hamiltonian even when $t = 0$, but will also include second order terms that arise from the contraction of two first order interaction vertices. 
Thus, the parameterization is

\begin{align}
    H^{(2)}(t) &= 2\pi g^2 \int\limits_{\substack{q_1, q_2\\ q_3, q_4}} \delta \left(q_1^+ + q_2^+ + q_3^+ + q_4^+ \right) B(t):\bar \psi(-q_1) \phi(q_2)\frac{\gamma^+}{q_3^+ + q_4^+} \phi(q_3) \psi(q_4): \\ \nonumber
    &+ g^2 \int\limits_{\substack{q_1, q_2, q_3\\ q_1', q_2', q_3'}} \delta \left(q_1^+ + q_2^+ + q_3^+ \right) \delta \left(q_1'^+ + q_2'^+ + q_3'^+ \right)B(t):C\left[\bar \psi(-q_1)\psi(q_2)\phi(q_3)\bar \psi(-q_1')\psi(q_2')\phi(q_3') \right]:.
\end{align}

Each second order interaction term has its own solution to equation \ref{eq:second-order}. 
The easiest interaction terms to solve are the instantaneous interaction terms. 
These can be solved in the same way the first order terms are solved, that is they pick up a form factor:

$$
H^{(2)}_{\text{inst.}}(t) = 2\pi g^2 \int\limits_{\substack{q_1, q_2\\ q_3, q_4}} \delta \left(q_1^+ + q_2^+ + q_3^+ + q_4^+ \right) e^{-t \left(q_1^+ + q_2^+ + q_3^+ + q_4^+ \right)^2}:\bar \psi(-q_1) \phi(q_2)\frac{\gamma^+}{q_3^+ + q_4^+} \phi(q_3) \psi(q_4):.
$$

The commutator between contracted $\mathcal{O}(g^2)$ terms in $H^{(2)}(t)$ and $H_0$ in equation \ref{eq:second-order} isn't as simple as the instantaneous terms, which follow the same procedure as section \ref{sec:first-order} (i.e. picking up a factor of $-\mathcal{Q}$, the sum of $q_i$'s into the vertices).

$\frac{dy}{dt} = -p(t)y(t) + q(t)$ has the general solution $y(t) = e^{-h(t)}\int e^{h(t)}q(t) + Ce^{-h(t)}$, where $h(t) = \int dt p(t)$