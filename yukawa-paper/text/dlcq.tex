\section{Discretized Lightcone Quantization (DLCQ)}
\label{sec:dlcq}

The main goal of the lightfront approach is calculating mass spectra. 
This is done by solving an eigenvalue equation \ref{eq:eigenvalue}.
In the continuum limit, this is intractible.
To obtain solutions, a discretization procedure is needed to computationally calculate $M^2$.
This procedure is known as discretized lightcone quantization (DLCQ).

\subsection{Continuous to discrete theory}
Any lightfront Hamiltonian ($H \equiv P^-$) written similarly to equation \ref{eq:yukawa-hamiltonian} can be cast as an eigenvalue equation of the form 
\begin{equation}
    \label{eq:eigenvalue}
    \left(\hat P^- \hat P^+ - \hat{\vec{P}}_\perp^2 \right)\ket{\psi} = M^2\ket{\psi}.
\end{equation}

In 1 + 1D, there is no transverse momentum, which simplifies this equation. 
In addition, we take $\ket{\psi}$ to be an eigenstate of $\hat P^+$ with eigenvalue $P^+$.
Thus, equation \ref{eq:eigenvalue} simplifies to 

\begin{equation}
    \label{eq:1p1Deigenvalue}
    \hat P^- \ket{\psi} = \frac{M^2}{P^+}\ket{\psi}.
\end{equation}

While $P^-$ is frame-dependent (it is simply a component of a four-vector), $P^+ P^-$ is a Lorentz scalar quantity. 
Thus, we can choose any value of $P^+$ for the total bound state longitudinal momentum of $\ket{\psi}$, diagonalize $\hat P^-$, and expect the same $M^2$ value (this is furthermore true due to boost-invariance in lightfront coordinates).

$\hat P^-$ (equation \ref{eq:yukawa-hamiltonian}) is an integral over all longitudinal partonic momentum. 
It doesn't know however about the value of $P^+$ assigned to the bound state. 
To remedy this, $\ket{\psi}$ must be expanded in a basis that explicitly knows about $P^+$.
This basis is chosen as $$\ket{\psi} = \sum_n \int d[\mu_n] \ket{\mu_n}\braket{\mu_n}{\psi}.$$
Here, $\{\ket{\mu_n} \}$ is a set of Fock states where $n$ labels the type of Fock state that makes up a given hadron $h$, i.e. in QCD, for a meson, these states would be $\{\ket{q\bar q}, \ket{q \bar qg}, \ket{gg}, \dots \}$
Furthermore, these Fock states contain information about $p_i^+$ and $\lambda_i$, the continuous partonic momentum and helicity of each parton in the Fock state. 
The integral over $d[\mu_n]$ ensures every value of momentum is in the Fock state such that the momentum sum for each state is equal to $P^+$.

The only tractable way to solve equation \ref{eq:1p1Deigenvalue} is by discretizing.
This is done by restricting $x^- \in [-L, L]$ (or placing our fields in a box). 
By doing this, the fields, which are in the continuum defined as 

\begin{align}
    \psi(x) &= \sum_\lambda \int_0^\infty \frac{dp^+}{4\pi p^+} \left(b_p u_\lambda(p) e^{-ip^+x^-} + d_p^\dagger v_\lambda(p) e^{ip^+x^-}\right)\\
    \bar \psi(x) &= \sum_\lambda \int_0^\infty \frac{dp^+}{4\pi p^+} \left(b_p^\dagger \bar u_\lambda(p) e^{ip^+x^-} + d_p \bar v_\lambda(p) e^{-ip^+x^-}\right)\\
    \phi(x) &= \int_0^\infty \frac{dp^+}{4\pi p^+}\left(a_p e^{-ip^+x^-} + a_p^\dagger e^{ip^+x^-}\right) \\ \nonumber
\end{align}

will necessarily be altered. 
Due to boundary conditions that now exist at $\pm L$, the integrals over continuous $p^+$ become sums over discrete $\bar p^+$.
The discrete fields are now given as \gus{ADD DISCRETE FIELDS}.

The momentum $p^+$ that a parton could take could be any continuous value such that it is $\in (0, P^+)$.
Now $p^+$ takes on discrete values given by
\begin{equation}
    \label{eq:discretemomentum}
    \bar p^+ = \frac{2\pi k}{L}; k = 
\begin{cases}
    \frac{1}{2},\frac{3}{2},\frac{5}{2} \dots ; \text{fermions}\\
    1, 2, 3, \dots ; \text{bosons}
\end{cases}\end{equation}

The continuum limit is recovered by taking $\pm L \rightarrow \pm \infty$.
By choosing a finite value of $L$, it necessarily restricts the space of states allowed, while also having the feature that increasing $L$ leads to a better approximation to the continuum limit.
To see this consider the following example.

Assume there is a hadron with $P^+ = 1$ with only one Fock state: $\ket{f\bar f}$.
Since the expansion of the bound state $\ket{\psi}$ integrates over all possible combinations of momentum for the fermion and antifermion, say that one basis state has $p_f^+ = 0.1, p_{\bar{f}}^+ = 0.9$.
To approximate this with some discrete $\bar p_f^+$ and $\bar p_{\bar f}^+$, choose an $L$, say $L = 2\pi$.
For this value of $L$, the only possible value of $k$ either $f$ or $\bar f$ can take is $\frac{1}{2}$.
This is because if we include $k = \frac{3}{2}$, then a state with $\bar p_f^+ = \frac{2\pi}{2\pi}\frac{1}{2}$ and $\bar p_{\bar f}^+ = \frac{2\pi}{2\pi}\frac{3}{2}$ has total longitudinal momentum $2$; however, we are in the $P^+ = 1$ sector, so this is not a valid state. 
Thus, for this value of $L$, the discrete momentum approximation of $p_f^+ = 0.1, p_{\bar{f}}^+ = 0.9$ is $\bar p_f^+ = \frac{2\pi}{2\pi}\frac{1}{2} = \frac{1}{2}$ and $\bar p_{\bar f}^+ = \frac{2\pi}{2\pi}\frac{1}{2} = \frac{1}{2}$, which is a poor approximation to their continuous values. 

Let us now increase $L$ and see how the approximation changes. 
Now take $L = 4\pi$. 
The possible values $k$ can take are now $k \in {\frac{1}{2}, \frac{3}{2}}$. 
Now we can obtain discrete values $\bar p_f^+ = \frac{2\pi}{4\pi}\frac{1}{2} = \frac{1}{4}$ and $\bar p_{\bar f}^+ = \frac{2\pi}{4\pi}\frac{3}{2} = \frac{3}{4}$, which is a slightly better approximation to $0.1$ and $0.9$ respectively. 

If we increase this logic, if we take $L = 10\pi$, then $k \in {\frac{1}{2}, \frac{3}{2}, \frac{5}{2}, \frac{7}{2}, \frac{9}{2}}$, and we can take $\bar p_f^+ = \frac{2\pi}{10\pi}\frac{1}{2} = \frac{1}{10}$ and $\bar p_{\bar f}^+ = \frac{2\pi}{10\pi}\frac{9}{2} = \frac{9}{10}$ which perfectly approximates the continuous values. 
If the continous values have a more complicated form (such as $p_f^+ = 0.13$, $p_{\bar f}^+ = 0.87$), then an even larger value of $L$ will be needed to better approximate these values. 

To sumarize this procedure, as $L$ is increased by integer multiplies of $2\pi$, this expands the space of states and leads to a better discrete approximation to the continuous partonic momentum. 
This can be written as taking $L = 2\pi K$, where $K \in \mathbb{Z}^+$. 
$K$ can be referred to as the \emph{resolution}, because by increasing $K$ leads to a finer momentum grid.
Thus, the limit $\pm L \rightarrow \pm \infty$ can be traded with a limit $K \rightarrow \infty$. 


\subsection{Fourier Transformation of Fields}

The fields are discretized in a 1+1D box of length $x^- \in \{-L, L\}$ which, due to boundary conditions and symmetry requirements for fermions and bosons, leads to discretized momenta, $p_k$ given in equation \ref{eq:discretemomentum}



For the scalar field, $\phi(x)$, $$\phi(x) = \sum_{k = 1}^\infty \frac{1}{\sqrt{4\pi k}}\left(a_{i_k} e^{-ip_k x} + a_{i_k}^\dagger e^{ip_k x} \right).$$
Here, the subscript $i_k$ refers to a particular index corresponding to mode $k$. 
We can transpose $\phi(x)$ to its corresponding momentum space field, denoted by $\tilde \phi(k)$: $$\tilde \phi(k) = \int_{-L}^L dx^- e^{\frac{i}{2}q_k^+ x^-}\phi(x).$$
The exponent in the exponential in this equation comes from the dot product in lightfront coordinates and asserting a fixed $x^+ = 0$ plane. 
Plugging in the form of $\phi(x)$, we obtain the discretized scalar field in momentum space as:

\begin{equation}
    \tilde \phi(k) = 2L \frac{\theta(k)a_{i_k} + \theta(-k)a_{-i_k}}{\sqrt{4\pi |k|}}.
\end{equation}

The fermionic field is $$\psi(x) = \sum_{k = 1/2}^\infty \frac{1}{\sqrt{4\pi k}}\left(b_{i_k}u(p_k) e^{-ip_k x} + v(p_k)d_{i_k}^\dagger e^{ip_k x} \right)$$ where $$u(p_k) = \frac{1}{\sqrt{p_k^+}}\left[\begin{matrix} p^+_k \\ m \end{matrix}\right],$$ $$v(p_k) = \frac{1}{\sqrt{p_k^+}}\left[\begin{matrix} -p^+_k \\ m \end{matrix}\right].$$
After the same Fourier transformation that leads to $\tilde \phi(x)$, we obtain 

\begin{equation}
    \tilde \psi(k) = 2L \frac{\theta(k)b_{i_k}u(p_k) + \theta(-k)d^\dagger_{-i_k}v(-p_k)}{\sqrt{4\pi |k|}}.
\end{equation}

From the Lagrangian, \ref{eq:yukawa-lagrangian}, we can write the stress-energy tensor and extract the Hamiltonian density: 

\begin{equation}
    \mathcal{H} = : \bar\psi \frac{\gamma^+}{2}
    \frac{ m^2 }
    {i\partial^+}\psi :
  + \frac{\mu^2}{2} :\phi \phi:
  + g : \bar\psi\psi : \phi
  + \frac{1}{2} g^2
  :\bar\psi \phi
  \frac{\gamma^+}{i\partial^+} \phi \psi:,
\end{equation}
which, after integrating over a spacetime 1+1D ``volume'', $dx^-$, we arrive at $$P^- = \int_{-L}^L dx^- \mathcal{H}.$$

From the discrete forms of the Fourier-transformed fields, we can write $\mathcal{H}$ in terms of $\tilde \phi(k)$ and $\tilde \psi(k)$.
\gus{Fill in derivation here.}

\begin{align}
    H_0 &= \sum_{k = 1/2}^\infty \frac{1}{k}\left(m_F^2 b_{i_k}^\dagger b_{i_k} + m_{\bar F}^2 d_{i_k}^\dagger d_{i_k} \right) + \sum_{k = 1}^\infty \frac{m_B^2}{k}a_{i_k}^\dagger a_{i_k}\\ \nonumber
    H_{3pt.} &= \\\nonumber
    H_{inst.} &= \\\nonumber
\end{align}

Now that we have a discrete form of the Hamiltonian, we must introduce a mode cutoff, $\Lambda$ to get a finite Hamiltonian.
This is accomplished by the substitution $\sum_k^\infty \rightarrow \sum_k^\Lambda$.
If one wanted to obtain the explicit matrix form of $H$, a basis of Fock states must be chosen. 
A protocol for choosing a basis is as follows:

\begin{enumerate}
    \item Choose a value of $P^+$.
    \item Construct all states of fermions, antifermions and bosons whose momentum sum to $1, 2, \dots, P^+$.
    \item Form a block of the Hamiltonian of fixed $P^+$ by calculating matrix elements $\langle i|H|j\rangle$.
\end{enumerate}

The matrix will now be block-diagonal, with each block having fixed $P^+$.
For a given block, the mode cutoff, $\Lambda$, is the same as the $P^+$ of the block.