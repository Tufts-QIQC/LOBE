\section{Discretized Lightcone Quantization (DLCQ)}
\label{sec:dlcq}

One can discretize equation \ref{eq:yukawa-hamiltonian} directly by discretizing each piece individually. 
Alternatively, and perhaps more straightforward, one may discretize the fields before writing out the Hamiltonian.
Also, since we want the Hamiltonian in momentum space, we perform a Fourier transformation of the discretized fields. 

The fields are discretized in a 1+1D box of length $x^- \in \{-L, L\}$ which, due to boundary conditions and symmetry requirements for fermions and bosons, leads to discretized momenta, $p_k$ given as 

$$p_k = \frac{2\pi k}{L}; k = 
\begin{cases}
    \frac{1}{2},\frac{3}{2},\frac{5}{2} \dots ; \text{fermions}\\
    1, 2, 3, \dots ; \text{bosons}
\end{cases}$$

For the scalar field, $\phi(x)$, $$\phi(x) = \sum_{k = 1}^\infty \frac{1}{\sqrt{4\pi k}}\left(a_{i_k} e^{-ip_k x} + a_{i_k}^\dagger e^{ip_k x} \right).$$
Here, the subscript $i_k$ refers to a particular index corresponding to mode $k$. 
We can transpose $\phi(x)$ to its corresponding momentum space field, denoted by $\tilde \phi(k)$: $$\tilde \phi(k) = \int_{-L}^L dx^- e^{\frac{i}{2}q_k^+ x^-}\phi(x).$$
The exponent in the exponential in this equation comes from the dot product in lightfront coordinates and asserting a fixed $x^+ = 0$ plane. 
Plugging in the form of $\phi(x)$, we obtain the discretized scalar field in momentum space as:

\begin{equation}
    \tilde \phi(k) = 2L \frac{\theta(k)a_{i_k} + \theta(-k)a_{-i_k}}{\sqrt{4\pi |k|}}.
\end{equation}

The fermionic field is $$\psi(x) = \sum_{k = 1/2}^\infty \frac{1}{\sqrt{4\pi k}}\left(b_{i_k}u(p_k) e^{-ip_k x} + v(p_k)d_{i_k}^\dagger e^{ip_k x} \right)$$ where $$u(p_k) = \frac{1}{\sqrt{p_k^+}}\left[\begin{matrix} p^+_k \\ m \end{matrix}\right],$$ $$v(p_k) = \frac{1}{\sqrt{p_k^+}}\left[\begin{matrix} -p^+_k \\ m \end{matrix}\right].$$
After the same Fourier transformation that leads to $\tilde \phi(x)$, we obtain 

\begin{equation}
    \tilde \psi(k) = 2L \frac{\theta(k)b_{i_k}u(p_k) + \theta(-k)d^\dagger_{-i_k}v(-p_k)}{\sqrt{4\pi |k|}}.
\end{equation}

From the Lagrangian, \ref{eq:yukawa-lagrangian}, we can write the stress-energy tensor and extract the Hamiltonian density: 

\begin{equation}
    \mathcal{H} = : \bar\psi \frac{\gamma^+}{2}
    \frac{ m^2 }
    {i\partial^+}\psi :
  + \frac{\mu^2}{2} :\phi \phi:
  + g : \bar\psi\psi : \phi
  + \frac{1}{2} g^2
  :\bar\psi \phi
  \frac{\gamma^+}{i\partial^+} \phi \psi:,
\end{equation}
which, after integrating over a spacetime 1+1D ``volume'', $dx^-$, we arrive at $$P^- = \int_{-L}^L dx^- \mathcal{H}.$$

From the discrete forms of the Fourier-transformed fields, we can write $\mathcal{H}$ in terms of $\tilde \phi(k)$ and $\tilde \psi(k)$.
\gus{Fill in derivation here.}

\begin{align}
    H_0 &= \sum_{k = 1/2}^\infty \frac{1}{k}\left(m_F^2 b_{i_k}^\dagger b_{i_k} + m_{\bar F}^2 d_{i_k}^\dagger d_{i_k} \right) + \sum_{k = 1}^\infty \frac{m_B^2}{k}a_{i_k}^\dagger a_{i_k}\\ \nonumber
    H_{3pt.} &= \\\nonumber
    H_{inst.} &= \\\nonumber
\end{align}

Now that we have a discrete form of the Hamiltonian, we must introduce a mode cutoff, $\Lambda$ to get a finite Hamiltonian.
This is accomplished by the substitution $\sum_k^\infty \rightarrow \sum_k^\Lambda$.
If one wanted to obtain the explicit matrix form of $H$, a basis of Fock states must be chosen. 
A protocol for choosing a basis is as follows:

\begin{enumerate}
    \item Choose a value of $P^+$.
    \item Construct all states of fermions, antifermions and bosons whose momentum sum to $1, 2, \dots, P^+$.
    \item Form a block of the Hamiltonian of fixed $P^+$ by calculating matrix elements $\langle i|H|j\rangle$.
\end{enumerate}

The matrix will now be block-diagonal, with each block having fixed $P^+$.
For a given block, the mode cutoff, $\Lambda$, is the same as the $P^+$ of the block.