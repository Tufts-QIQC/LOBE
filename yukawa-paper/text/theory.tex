\section{Lightfront Yukawa Theory}
\label{sec:yukawa-theory}
We start with the Lagrangian of the Yukawa model:
\begin{equation}
    \label{eq:yukawa-lagrangian}
    \mathcal{L}_{Yukawa} = \bar \psi \left(i\gamma^\mu \partial_\mu - m \right)\psi + \frac{1}{2}\partial_\mu \phi \partial^\mu \phi - \frac{1}{2}\mu^2\phi^2 - g\bar \psi \psi \phi
\end{equation}

where $m$ is the fermion mass, and $\mu$ is the boson mass.
The Hamiltonian density can be obtained from the Lagrangian by the $+-$ component of $T^{\mu \nu}$, the energy-momentum tensor.
The Hamiltonian is the integral of the corresponding density, integrated over a volume, which in 1 + 1D lightfront coordinates is:

\begin{equation}
    H = \int dx^- \mathcal{H}.
\end{equation}

This leads to a Hamiltonian written as a sum of three terms: 

\begin{equation}
    \label{eq:Ham-terms}
    H = H_0 + H_Y + H_I,
\end{equation}
where $H_0$ is the free kinetic energy of the particles, $H_Y$ represents the standard Yukawa 3-point interaction vertices, and $H_I$ is the \textit{instantaneous} interaction term, existent in lightfront coordinates, but not instant-time.

Equation \ref{eq:Ham-terms} can be written in a momentum space representation, which leads to easier calculations.
Explicitly, these Hamiltonian are given below:


\begin{equation}
    \label{eq:H-free}
    H_0 = \frac{m^2}{2}\int \frac{dq^+}{4\pi} :\bar \psi(q) \frac{\gamma^+}{q^+}\psi(q): + \frac{\mu^2}{2}\int \frac{dq^+}{4\pi} :\phi(-q)\phi(q): 
\end{equation}
\begin{equation}
    \label{eq:H-3}
    H_Y = 4\pi g\int \frac{dq_1^+ dq_2^+ dq_3^+}{(4\pi)^3}\delta(q_1^+ + q_2^+ +q_3^+):\bar \psi(-q_1) \psi(q_2) \phi(q_3):
\end{equation}
\begin{equation}
    \label{eq:H-4}
    H_I = 2\pi g^2\int \frac{dq_1^+ dq_2^+ dq_3^+ dq_4^+}{(4\pi)^4}\delta(q_1^+ + q_2^+ +q_3^+ + q_4^+):\bar \psi(-q_1) \phi(q_2)\frac{\gamma^+}{q_3^+ + q_4^+} \phi(q_3) \psi(q_4):
\end{equation}
The continuous momentum space fields are:

\begin{equation}
    \label{eq:phi}
    \phi(q) = \frac{\theta(q^+)a_q + \theta(-q^+)a_{-q}^\dagger}{|q^+|}
\end{equation}
\begin{equation}
    \label{eq:psi}
    \psi(q) = \frac{\theta(q^+)u(q)b_q + \theta(-q^+)v(-q)d_{-q}^\dagger}{|q^+|}\\
\end{equation}


The discretization procedure used for this Hamiltonian is Discretized Lightcone Quantization (DLCQ) \cite{}. 
The procedure is explained in detail in appendex \ref{sec:dlcq}.