\documentclass[%
 reprint,
%superscriptaddress,
%groupedaddress,
%unsortedaddress,
%runinaddress,
%frontmatterverbose, 
%preprint,
%preprintnumbers,
nofootinbib,
%nobibnotes,
%bibnotes,
 amsmath,amssymb,
 aps,
%pra,
%prb,
%rmp,
%prstab,
%prstper,
%floatfix,
]{revtex4-2}

\usepackage[utf8]{inputenc} % allow utf-8 input
\usepackage[T1]{fontenc}    % use 8-bit T1 fonts
\usepackage{hyperref}       % hyperlinks
\usepackage{url}            % simple URL typesetting
\usepackage{booktabs}       % professional-quality tables
\usepackage{amsfonts}       % blackboard math symbols
\usepackage{nicefrac}       % compact symbols for 1/2, etc.
\usepackage{microtype}      % microtypography
\usepackage{lipsum}
\usepackage{fancyhdr}       % header
\usepackage{blindtext}
\usepackage{physics}
\usepackage{amssymb,amsmath}
\numberwithin{equation}{section}
\usepackage{bbm}
\usepackage{listings}
\usepackage[dvipsnames]{xcolor}

\bibliographystyle{apsrev4-2}




\usepackage{graphicx}% Include figure files
\usepackage{dcolumn}% Align table columns on decimal point
\usepackage{bm}% bold math
%\usepackage{hyperref}% add hypertext capabilities
%\usepackage[mathlines]{lineno}% Enable numbering of text and display math
%\linenumbers\relax % Commence numbering lines

%\usepackage[showframe,%Uncomment any one of the following lines to test 
%%scale=0.7, marginratio={1:1, 2:3}, ignoreall,% default settings
%%text={7in,10in},centering,
%%margin=1.5in,
%%total={6.5in,8.75in}, top=1.2in, left=0.9in, includefoot,
%%height=10in,a5paper,hmargin={3cm,0.8in},
%]{geometry}

\begin{document}

\preprint{APS/123-QED}

\title{A Block-Encoding of the Renormalized Quartic Oscillator Hamiltonian}% Force line breaks with \\

\author{Gustin, Serafin, Simon, Goldstein, Love}%
 %\email{carter.gustin@tufts.edu}

\affiliation{Tufts University}%



\date{\today}% It is always \today, today,
             %  but any date may be explicitly specified

\begin{abstract}
Abstract
\end{abstract}

%\keywords{Suggested keywords}%Use showkeys class option if keyword
                              %display desired
\maketitle

\section{Introduction}
\subsection{The Quartic Oscillator Model} 
The quartic oscillator can be viewed as a quantum field theory as $\phi^4$ constrained to
one point in space time. Althernatively, it can be viewed as an extension of the quantum
simple harmonic oscillator with a quartic potential term added, e.g.:
\begin{equation}
  H = \frac{\dot{\phi}^2}{2} + \frac{m^2\phi^2}{2} + gm^3\phi^4.
\end{equation}

In second quantized form, the Hamiltonian can be expressed in terms of creation and 
annihilation operators as

\begin{equation}
  \label{Hamiltonian}
  H = a^\dagger a + g(a + a^\dagger)^4
\end{equation}
where $m = 1$, and $g$ is an arbitrary coupling constant. The normalized eigenstates of
the free part of the Hamiltonian, $H_0$ are $\ket{k} = (k!)^{-1/2}(a^\dagger)^k\ket{0}$. The goal
is thus, given this basis, finding the ground state eigenvalue of the problem when the 
quartic potential is turned on ($g \neq 0$). 

In the basis of states defined above, the Hamiltonian matrix, 
$H_{kl} \equiv\matrixel{k}{H}{l}$ is 5-band-diagonal. We first must pick some cutoff
on the basis states to get an explicit matrix form. In general, we want this cutoff, $N$,
to be very large, such that diagonalizing this $N \times N$ Hamiltonian gives a very good 
approximation to the true ground state. Note that in this model, when it is said that
there is no divergence, it means that as $N \rightarrow \infty$, the ground state energy
$E_0$ tends to the true ground state of the infinite problem. In more complicated field
theories, this won't be true and will require more advanced renormalization techniques.

This $N \times N$ matrix can be diagonalized exactly, or simulated to give an approximation to
$E_0$; however, this will be very expensive. \textcolor{red}{We should be explicit in 
what we mean by expensive?} Alternatively, we can ``renormalize'' the model, allowing quantum
simulation to be performed. This is done with Gaussian elimination.

\section{Renormalization}
\subsection{Gaussian Elimination}
The quartic oscillator is a model that is not studied here within the context
of quantum field theories; however, we can utilize techniques from these 
theories to reduce the resources needed to simulate this model. 
In quantum field theories, renormalization is the process of removing infinities
that arise when doing calculations beyond leading order. Here, Gaussian elimination 
is performed to systematically remove one row and column from the ``infinitely''
large Hamiltonian, thus reducing the dimension of the matrix. Although there are no 
divergences that arise in this model, Gaussian elimination still is useful. This is done
approximately, since pure Gaussian elimination isn't any cheaper than 
diagonalizing the infinite matrix. 

Guassian elimination starts by writing down the eigenvalue equation: 

\begin{equation}
  H\psi^{(0)} = E_0 \psi^{(0)}
\end{equation}
where $\psi^{(0)}$ is a $N \times 1$ vector with entries $\psi_0, \dots \psi_{N - 1}$ and $H$
is a $N \times N$ matrix. Writing $H\psi = E\psi$ for the ground state in matrix form gives 
$N$ coupled equations. To perform Gaussian elimination, one writes down the equation
corresponding to $\psi^{(0)}_{N - 1}$:

\begin{equation}
  \sum_{l = 0}^{N - 1} H_{N - 1, l}\psi_l = E_0\psi_{N - 1}
\end{equation}
From here, we rewrite this equation in terms of $\psi_{N - 1}$ and plug this into the equation
for $\psi^{(0)}_{N - 2}$. We say this removes a row and column from the matrix because now
there is one less independent degree of freedom. The resulting set of equations is now given
as:

\begin{equation}
  \sum_{l = 0}^{N - 2}\left(H_{N-2, l} + \frac{H_{N-2, N}H_{N, l}}{E_0 - H_{N, N}} \right)\psi_{N - 2} = E_0\psi_{N - 2}
\end{equation}

A problem now becomes evident: the eigenvalue equation for $\psi_{N - 2}$ now depends on $E_0$ in two places.
This is where an approximation comes in that allows us to ``approximately'' Gaussian eliminate the
Hamiltonian matrix, leading to a recurrsion relation that shrinks the size of the matrix. We assume
that the ground state eigenvalue is small compared to the bottom-right-most matrix element (which 
is the large-energy corner). Thus, we can let
\begin{equation}
  E_0 - H_{N, N} \approx -H_{N, N}
  \label{approx}
\end{equation}

Now, a recurrsion relation can be defined on each matrix element $H_{kl}$ where the ``new'' matrix
after one Gaussian elimination step is one row and column smaller:

\begin{equation}
  \label{recurrsion}
  H_{kl}^{(j)} = H_{kl}^{(j - 1)} - g\frac{H_{k,N-j}^{(j - 1)}H_{N-j,l}^{(j - 1)}}{H_{N-j,N-j}^{(j - 1)}}
\end{equation}

There comes a point when using this recurssion relation where this approximation in \ref{approx} starts to break down. This becomes obvious when examining the spectrum of the matrix with a large cutoff $N \gg 0$ vs. the matrix after $k = N - n$ renormalization steps in figure \ref{qosc_renorm_eigvals}.
Note that there is a linear threshold on this figure at $10^{-9}$ to remove noise for very small energy differences at larger $n$. 
Another take

For the purposes of this work, figure \ref{qosc_renorm_eigvals} shows that in renormalizing the quartic oscillator Hamiltonian down to a $4 \times 4$ Hamiltonian, the ground state eigenvalue will be accurate up to $1\%$ error. 

\begin{figure}
  \label{qosc_renorm_eigvals}
  \includegraphics[width = \linewidth]{figures/qosc_renorm_eigvals.png}
  \caption{Difference between the renormalized Hamiltonian spectrum and the continuum limit Hamiltonian spectrum}
\end{figure}

\subsection{Renormalized Hamiltonian}
In order to block-encode this Hamiltonian with the Ladder Operator Block Encoding (LOBE) algorithm, it is necessary to get an explicit expression for the renormalized Hamiltonian in terms of creation and annihilation operators. 
This will be done by utilizing the recurssion relation in equation \ref{recurrsion}. For one Gaussian elimination step, going from a $N \times N \rightarrow (N - 1)\times (N - 1)$ dimension Hamiltonian, the matrix elements are updated via equation \ref{recurrsion}. Due to the band-structure of the Hamiltonian, one can see that the only matrix elements that will be updated in one Gaussian elimination step are $H_{N-4, N-4}, H_{N-2, N-2}, H_{N-4, N-2}, H_{N-2, N-4}$. 

Certain combinations of ladder operators can be used to modify the 4 matrix elements in each Gaussian elimination step. To modify the matrix element $H_{N-4, N-4}$, one needs to add an operator of the form $C_{N-4, N-4}(a^\dagger)^{N-4}a^{N-4}$, while for $H_{N-2, N-2}$, one needs to add an operator of the form $C_{N-2, N-2}(a^\dagger)^{N-2}a^{N-2}$, and similarly for $H_{N-4, N-2}, H_{N-2, N-4}$, add $C_{N-4, N-2}(a^\dagger)^{N-4}a^{N-2}, C_{N-2, N-4}(a^\dagger)^{N-2}a^{N-4}$.

In adding these terms, other matrix elements will be affected. This necessitates counterterms to be added to undo unwanted affects. 

\section{Simulation of Renormalized Hamiltonian via LOBE}
\subsection{Overview of LOBE}
\textcolor{Green}{Will}

\subsection{Resource Estimates}
\textcolor{red}{What are LOBE estimates for the ``infinite'' Hamiltonian matrix vs. the 
renormalized matrix. How do we quantify if this trade-off is useful?}

\section{Conclusion}



\appendix
\section{Renormalized Hamiltonian in terms of Ladder Operators}
The following terms must be added to the Hamiltonian in equation \ref{Hamiltonian}, in order to produce a renormalized Hamiltonian in terms of ladder operators. Note that $n$ is the size of the renormalized matrix such that $n = N - k$, with $k$ Gaussian elimination steps.
What follows is the expression for the renormalized Hamiltonian, where each term (product of ladder operators) has a coefficient of the form $\Delta H(i,j)$ which picks up the information about the recurssion relation \ref{recurrsion}:

\begin{widetext}
\begin{align}
  \label{renH}
  H_{eff} = H &+ \frac{\Delta H(n-3, n-3)}{(n-3)!}(a^\dagger)^{n-3}a^{n-3} - \frac{\Delta H(n-3, n-3)}{(n-3)!}(a^\dagger)^{n-2}a^{n-2} + \frac{\Delta H(n-3, n-3)}{2(n-3)!}(a^\dagger)^{n-1}a^{n-1}\\ \nonumber
             &- \frac{\Delta H(n-3, n-3)}{6(n-3)!}(a^\dagger)^{n}a^{n} +  \frac{\Delta H(n-1, n-1)}{(n-1)!}(a^\dagger)^{n - 1}a^{n - 1} - \frac{\Delta H(n-1, n-1)}{(n-1)!}(a^\dagger)^{n}a^{n}\\ \nonumber
             &+ \frac{\Delta H(n-3, n-1)}{\sqrt{(n-3)!(n-1)!}}\left[(a^\dagger)^{n-3}a^{n-1} - (a^\dagger)^{n-2}a^{n} \right]+ \frac{\Delta H(n-1, n-3)}{\sqrt{(n-1)!(n-3)!}}\left[(a^\dagger)^{n-1}a^{n-3} - (a^\dagger)^{n}a^{n-2} \right] \\ \nonumber
             &+ \frac{\Delta H(n-2, n-2)}{(n-2)!}(a^\dagger)^{n-2}a^{n-2} - \frac{\Delta H(n-2, n-2)}{(n-2)!}(a^\dagger)^{n-1}a^{n-1} + \frac{\Delta H(n-2, n-2)}{2(n-2)!}(a^\dagger)^{n}a^{n} \\ \nonumber
             &-\frac{\Delta H(n-2, n-2)}{6(n-2)!}(a^\dagger)^{n+1}a^{n+1} + \frac{\Delta H(n, n)}{n!}(a^\dagger)^n a^n -\frac{\Delta H(n, n)}{n!}(a^\dagger)^{n + 1} a^{n + 1} \\ \nonumber
             &+ \frac{\Delta H(n-2, n)}{\sqrt{(n)!(n-2)!}}\left[(a^\dagger)^{n-2}a^{n} - (a^\dagger)^{n-1}a^{n+1} \right]+ \frac{\Delta H(n, n-2)}{\sqrt{(n)!(n-2)!}}\left[(a^\dagger)^{n}a^{n-2} - (a^\dagger)^{n+1}a^{n-1} \right]
\end{align}
\end{widetext}
where $H$ is the model Hamiltonian given in equation \ref{Hamiltonian}.

To obtain the proper coefficients, $\Delta H(i,j)$, one must first generate a matrix from the relation \ref{recurrsion} of size $n \times n$. From here, it is possible to obtain the proper coefficients for equation \ref{renH} to obtain:

\begin{equation}
  \Delta H(i, j) = \tilde{H}(i,j) - H(i,j)
\end{equation}
where $\tilde{H}(i,j)$ is the matrix element from the recurrsion relation.
\end{document}