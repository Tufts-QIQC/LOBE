\documentclass[%
 reprint,
%superscriptaddress,
%groupedaddress,
%unsortedaddress,
%runinaddress,
%frontmatterverbose, 
%preprint,
%preprintnumbers,
nofootinbib,
%nobibnotes,
%bibnotes,
 amsmath,amssymb,
 aps,
%pra,
%prb,
%rmp,
%prstab,
%prstper,
%floatfix,
]{revtex4-2}

\usepackage[utf8]{inputenc} % allow utf-8 input
\usepackage[T1]{fontenc}    % use 8-bit T1 fonts
\usepackage{hyperref}       % hyperlinks
\usepackage{url}            % simple URL typesetting
\usepackage{booktabs}       % professional-quality tables
\usepackage{amsfonts}       % blackboard math symbols
\usepackage{nicefrac}       % compact symbols for 1/2, etc.
\usepackage{microtype}      % microtypography
\usepackage{lipsum}
\usepackage{fancyhdr}       % header
\usepackage{blindtext}
\usepackage{physics}
\usepackage{amssymb,amsmath}
\numberwithin{equation}{section}
\usepackage{bbm}
\usepackage{listings}

\bibliographystyle{apsrev4-2}




\usepackage{graphicx}% Include figure files
\usepackage{dcolumn}% Align table columns on decimal point
\usepackage{bm}% bold math
%\usepackage{hyperref}% add hypertext capabilities
%\usepackage[mathlines]{lineno}% Enable numbering of text and display math
%\linenumbers\relax % Commence numbering lines

%\usepackage[showframe,%Uncomment any one of the following lines to test 
%%scale=0.7, marginratio={1:1, 2:3}, ignoreall,% default settings
%%text={7in,10in},centering,
%%margin=1.5in,
%%total={6.5in,8.75in}, top=1.2in, left=0.9in, includefoot,
%%height=10in,a5paper,hmargin={3cm,0.8in},
%]{geometry}

\begin{document}

\preprint{APS/123-QED}

\title{Working Title: Block Encoding the Renormalized Quartic Oscillator Hamiltonian}% Force line breaks with \\

\author{Gustin, Serafin, Simon, Goldstein, Love}%
 %\email{carter.gustin@tufts.edu}

\affiliation{Tufts University}%



\date{\today}% It is always \today, today,
             %  but any date may be explicitly specified

\begin{abstract}
Abstract
\end{abstract}

%\keywords{Suggested keywords}%Use showkeys class option if keyword
                              %display desired
\maketitle

\section{Introduction}
\subsection{The Quartic Oscillator Model} 

\section{Renormalization}
The quartic oscillator is a model that is not studied here within the context
of quantum field theories; however, we can utilize techniques from these 
theories to reduce the resources needed to simulate this model. 
In quantum field theories, Renormalization is the process of removing infinities
that arise when doing calculations beyond leading order. Here, Gaussian elimination 
is performed to systematically remove one row and column from the "infinitely"
large Hamiltonian, thus reducing the dimension of the matrix. This is done
approximately, since pure Gaussian elimination isn't any cheaper than 
diagonalizing the infinite matrix. 

Guassian elimination starts by writing down the eigenvalue equation for the 
largest energy eigenvalue of the Hamiltonian. 
\subsection{Renormalized Hamiltonian}

\section{Simulation of Renormalized Hamiltonian via LOBE}
\subsection{Overview of LOBE}

\section{Conclusion}

\end{document}